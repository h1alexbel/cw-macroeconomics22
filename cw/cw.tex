% #MIT License
% #
% #Copyright (c) 2022 Aliaksei Bialiauski
% #
% #Permission is hereby granted, free of charge, to any person obtaining a copy
% #of this software and associated documentation files (the "Software"), to deal
% #in the Software without restriction, including without limitation the rights
% #to use, copy, modify, merge, publish, distribute, sublicense, and/or sell
% #copies of the Software, and to permit persons to whom the Software is
% #furnished to do so, subject to the following conditions:
% #
% #The above copyright notice and this permission notice shall be included in all
% #copies or substantial portions of the Software.
% #
% #THE SOFTWARE IS PROVIDED "AS IS", WITHOUT WARRANTY OF ANY KIND, EXPRESS OR
% #IMPLIED, INCLUDING BUT NOT LIMITED TO THE WARRANTIES OF MERCHANTABILITY,
% #FITNESS FOR A PARTICULAR PURPOSE AND NONINFRINGEMENT. IN NO EVENT SHALL THE
% #AUTHORS OR COPYRIGHT HOLDERS BE LIABLE FOR ANY CLAIM, DAMAGES OR OTHER
% #LIABILITY, WHETHER IN AN ACTION OF CONTRACT, TORT OR OTHERWISE, ARISING FROM,
% #OUT OF OR IN CONNECTION WITH THE SOFTWARE OR THE USE OR OTHER DEALINGS IN THE
% #SOFTWARE.

\documentclass[14pt,a4paper]{article}

\usepackage[a4paper, left={1.35in}, outer={0.55in}, top={1in}, bottom={1in}]{geometry}
\usepackage[utf8]{inputenc}
\usepackage[T1]{fontenc}
\usepackage{amsmath}
\usepackage[russian]{babel}
\usepackage{lipsum}

\addto\captionsrussian{
    \renewcommand{\contentsname}%
    {СОДЕРЖАНИЕ}%
}

\newenvironment{linez}
{\trivlist\nopagebreak
\parindent0pt
\item\relax\obeylines}
{\par
\vspace{3pt}%
\endtrivlist}

\begin{document}
    \begin{titlepage}
        \newcommand{\HRule}{\rule{\linewidth}{0mm}}
        \center
        \textsc{\Large МИНИСТЕРСТВО ОБРАЗОВАНИЯ РЕСПУБЛИКИ БЕЛАРУСЬ}\\[0cm]
        \renewcommand{\baselinestretch}{1.5}
        \textsc{\Large УО «БЕЛОРУССКИЙ ГОСУДАРСТВЕННЫЙ ЭКОНОМИЧЕСКИЙ УНИВЕРСИТЕТ»}\\[2cm]
        \textsc{ Кафедра экономической политики}\\[0.5cm]
        \HRule\\[1cm]
        {\bfseries КУРСОВАЯ РАБОТА}\\[0.4cm]
        \HRule\\[0cm]
        \begin{linez}
            \\ по дисциплине: {\bfseries Макроэкономика}
            \\ на тему: {\bfseries Социально-ориентированная рыночная экономика: черты, принципы и направления. Особенности белорусской модели развития.}
        \end{linez}
        \HRule\\[2cm]

        \begin{minipage}{0.4\textwidth}
            \begin{flushleft}
                \begin{tabular*}{\textwidth}{@{}p{.45\textwidth}@{\extracolsep{\fill}}r@{}}
                    \raggedright
                    \large{Студент}
                    \vspace{-5pt}
                    \\ФЭМ, 2-й курс, 21-ДКС
                \end{tabular*}
            \end{flushleft}
        \end{minipage}
        ~
        \begin{minipage}{0.4\textwidth}
            \begin{flushleft}
                \hspace{1.5cm}
                А.В. \textsc{Белявский}
            \end{flushleft}
        \end{minipage}
        \HRule\\[0.5cm]
        \begin{minipage}{0.4\textwidth}
            \begin{flushleft}
                \large{Руководитель}\\
                \vspace{-5pt}
                кандидат экон.наук, доцент
            \end{flushleft}
        \end{minipage}
        ~
        \begin{minipage}{0.4\textwidth}
            \begin{flushleft}
                \hspace{1.5cm}
                A.Д. \textsc{Якутович}
            \end{flushleft}
        \end{minipage}

        \HRule\\[5.5cm]
        {\large МИНСК 2022}
        \vfill
    \end{titlepage}

    \newpage
    \begin{center}
        \textbf{\Large{РЕФЕРАТ}}
    \end{center}
    \par
    Курсовая работа: ?с., ? табл., ? источник.
    \\
    \par
    \large{СОЦИАЛЬНО-ОРИЕНТИРОВАННАЯ ЭКОНОМИКА, ЛИБЕРАЛЬНЫЙ РЕЖИМ, КОНСЕРВАТИВНЫЙ РЕЖИМ, СОЦИАЛ-ДЕМОКРАТИЧЕСКИЙ РЕЖИМ,}
    \\
    \large{СОЦИАЛЬНО-ЭКОНОМИЧЕСКОЕ РАЗВИТИЕ, РУДИМЕНТАРНЫЙ РЕЖИМ}
    \\
    \par
    \textbf{Объект исследования - } Социально-ориентированная рыночная экономика.
    \par
    \textbf{Предмет исследования - } модель социально-экономического развития Республики Беларусь.
    \par
    \textbf{Цель работы:} выявить условия формирования социально-ориентированной экономики, проанализировать модель социально-ориентированной рыночной экономики в Республике Беларусь, определить проблемы развития социально-ориентированной экономики республики и соответствующих перспектив.
    \par
    \textbf{Методы исследования:} метод описания, сравнительного анализа, системного подхода, исторический метод.
    \par
    \textbf{Исследования и разработки:} проведен анализ модели социально-ориентированной экономики в Республике Беларусь, в результате которого рассмотрены особенности ее реализации и пути совершенствования в Республике Беларусь
    \par
    \textbf{Элементы научной новизны:} определены и сформулированы основные проблемы модели социально-ориентированной экономики в Республике Беларусь и предложены способы их решения.
    \par
    \textbf{Область возможного практического применения:} результаты, полученные в курсовой работе, могут быть использованы на лекционных и семинарских занятиях по изучению курса «макроэкономика».
    \\
    \\
    \tabcolsep365pt
    \hfill\begin{tabular}{lp{.5\linewidth}@{}}
              \_\_\_\_\_\_\_\_\_\_
              \\
              (подпись студента) \\
    \end{tabular}

    \newpage
    \begin{center}
        \textbf{\LARGE{ESSAY}}
    \end{center}
    \par
    Course work: ?p., ? tab., ? sources.
    \\
    \par
    \large{SOCIO-ORIENTED ECONOMY, LIBERAL MODEL, CONSERVATIVE MODEL,}
    \\
    \large{SOCIAL-DEMOCRATIC MODEL, SOCIO-ECONOMIC DEVELOPMENT,}
    \\
    \large{RUDIMENTARY MODEL}
    \\
    \par
    \textbf{The object of study} is a socially-oriented market economy.
    \par
    \textbf{The subject of study} is a model of socio-economic development of the Republic of Belarus.
    \par
    \textbf{Purpose of the work:} to identify the conditions for the formation of a socially oriented economy, analyze the model of a socially oriented market economy in the Republic of Belarus, identify the problems of development of a socially oriented economy of the republic and the corresponding prospects.
    \par
    \textbf{Research methods:} method of description, comparative analysis, systematic approach, historical method.
    \par
    \textbf{Research and development:} an analysis of the model of a socially oriented economy in the Republic of Belarus was carried out, as a result of which the features of its implementation and ways of improvement in the Republic of Belarus were considered.
    \par
    \textbf{Elements of scientific novelty:} the main problems of the model of a socially oriented economy in the Republic of Belarus are identified and formulated, and ways to solve them are proposed.
    \par
    \textbf{Area of possible practical application:} the results obtained in the course work can be used in lectures and seminars on the study of the course ``macroeconomics''.
    \\
    \\
    \tabcolsep365pt
    \hfill\begin{tabular}{lp{.5\linewidth}@{}}
              \_\_\_\_\_\_\_\_\_\_\_\_\_\_
              \\
              (signature of the student) \\
    \end{tabular}

    \newpage
    \begin{center}
        \fontsize{10}{}
        \tableofcontents
        \addcontentsline{toc}{section}{РЕФЕРАТ}
        \addcontentsline{toc}{section}{ВВЕДЕНИЕ}
        \addcontentsline{toc}{section}{1 Социально-ориентированная рыночная экономика(СОРЭ)}
        \addcontentsline{toc}{section}{1.1 Определение и ключевые понятия}
        \addcontentsline{toc}{section}{1.2 Главные компоненты СОРЭ}
        \addcontentsline{toc}{section}{1.3 Черты и принципы СОРЭ}
        \addcontentsline{toc}{section}{1.4 Роль государства в СОРЭ}
        \addcontentsline{toc}{section}{1.5 Недостатки СОРЭ}
        \addcontentsline{toc}{section}{2 Опыт, модели и направления СОРЭ в других странах}
        \addcontentsline{toc}{section}{2.1 Модели и направления СОРЭ}
        \addcontentsline{toc}{section}{2.2 СОРЭ в Швеции}
        \addcontentsline{toc}{section}{2.3 СОРЭ в Германии}
        \addcontentsline{toc}{section}{3 Направления СОРЭ в РБ}
        \addcontentsline{toc}{section}{3.1 Социально-экономическое развитие как направление СОРЭ в РБ}
        \addcontentsline{toc}{section}{3.2 Основные проблемы социально-экономического развития РБ}
        \addcontentsline{toc}{section}{ЗАКЛЮЧЕНИЕ}
        \addcontentsline{toc}{section}{СПИСОК ИСПОЛЬЗОВАННЫХ ИСТОЧНИКОВ}
    \end{center}

    \newpage
    \begin{center}
        \textbf{\LARGE{ВВЕДЕНИЕ}}
    \end{center}
    \\
    \par
    Девяностые годы XX века были судьбоносными для экономики Беларуси. Распался Советский Союз, рухнули прежние принципы организации и функционирования экономики, практически все бывшие союзные республики оказались в глубоком политическом и экономическом кризисе.
    \par
    Именно становление экономической системы рыночного типа выдвигает в число первоочередных такие вопросы, как преобразование механизма управления, изменение принципов мотивации труда. Актуальность этих вопросов для Республики Беларусь вызвана сложностью переходного периода, выбором пути социально-экономического развития, кризисом всех сфер общества, необходимостью создания собственной модели экономики.
    \par
    Возникла необходимость в сжатые сроки создать собственную государственность, национальную банковскую систему, ввести национальную валюту, искоренить гиперинфляцию, насытить рынок элементарными потребительскими товарами, поддерживать отечественных производителей, сдерживать рост безработицы, защитить социально наиболее уязвимые слои населения, определить перспективную модель социально-экономического развития страны, овладеть навыками конкурирования на мировых рынках товаров и услуг.
    \par
    Актуальность темы исследования курсовой работы связана с необходимостью определения целей и условий формирования социально - ориентированной рыночной экономики, опираясь на опыт других стран, в определении проблем и перспектив социально-экономического развития Республики Беларусь. Анализ этих вопросов даст не только четкую оценку идеям, использовавшимся при формировании хозяйства нашей Республики, но и поможет определить и сформулировать новые, наиболее созвучные времени и мировым тенденциям развития экономики государства.
    \par
    Главной целью курсовой работы является необходимость понять условия формирования социально-ориентированной экономики, рассмотреть модель социально-ориентированной рыночной экономики в нашей стране, определить проблемы ее развития и соответствующие перспективы.
    \par
    Задачи курсовой работы:
    \begin{itemize}
        \item выявить понятие и сущность социально-ориентированной экономики;
        \item рассмотреть основные черты и модели социально-ориентированной экономики;
        \item проанализировать модель экономического развития Республики Беларусь и ее дальнейшие перспективы.
    \end{itemize}
    \par
    В ходе написания курсовой работы использовались учебные пособия, нормативные документы, статистические данные, интернет источники.

    \newpage
    \begin{center}
        \textbf{1 Социально-ориентированная рыночная(СОР) экономика}
        \\
        \HRule\\[0.5cm]
        \textbf{1.1 Определение и ключевые понятия}
    \end{center}
    \\
    \par
    Есть две противоположных экономических модели: плановая и рыночная экономика. Подавляющее большинство стран сейчас развиваются по рыночной модели.
    \par
    Социа́льно-ориенти́рованная ры́ночная эконо́мика (социа́льное ры́ночное хозя́йство) (нем. Soziale Marktwirtschaft) — экономическая система, организованная на основе рыночной саморегуляции, при которой координация действий осуществляется на основе взаимодействия на рынках свободных частных производителей и свободных индивидуальных потребителей. Модель социально-рыночной экономики исходит из требования, что ни государство, ни частный бизнес не вправе иметь полный контроль над экономикой, а должны служить людям. В этой разновидности смешанной экономики, так же как и в рыночной экономике, структура распределения ресурсов определяется исключительно решениями самих потребителей, поставщиков ресурсов и частных фирм. Однако при этом экономически более сильные обязаны поддерживать более слабых. Роль государства заключается в развитии чувства взаимной ответственности всех участников на рынке и в корректировке несправедливых тенденций в конкуренции, торговле и распределении доходов. Система рассматривалась как альтернатива «laissez-faire» капитализму и социализму.
    \par
    Концепцию начал реализовывать Людвиг Эрхард, министр экономики, а впоследствии федеральный канцлер Германии. Название системы социально-рыночной экономики дал в 1947 году экономист Альфред Мюллер-Армак. Важный вклад в развитие также внесли Франц Бёмruen, Вальтер Ойкен, Франц Оппенгеймер, Вильгельм Рёпке, Александер Рюстов, Жак Фреско, Концепция стала одним из существенных элементов идеологии христианско-демократического движения и получила поддержку со стороны социальных консерваторов, социальных либералов и социал-демократов.
    \par
    Термин «социально-ориентированная экономика» является в последнее время достаточно употребляемым, хотя в обыденном токовании представление о нем достаточно расплывчато. Не совсем понято, является ли ориентация на высокие социальные параметры характеристикой всех экономик, или она присуща только некоторым разновидностям. Как известно, любая экономическая система имеет своей целью достижение высокого уровня удовлетворения материальных и духовных потребностей людей, а потому обладает социальными моментами. Действительно, не может же экономика развиваться ради себя самой: конечным результатом и целью жизнедеятельности является человек. Тем не менее, под социально-ориентированной понимается такой тип экономической системы, который отвечает определенным критериям. Можно толковать социально-ориентированную экономику как систему, в которой наиболее значимой является перераспределительная функция государства, позволяющая устранить существенную дифференциацию доходов в обществе и ликвидировать полюса бедности и богатства. Данное определение будет не совсем точным и правильным, поскольку любое государство занимается перераспределением доходов посредством функционирования бюджетной системы, поэтому вряд ли можно говорить о социальности каждой экономики.
    \par
    Социализация экономики стала реальностью с 20 века. Многие исследователи признают ее настолько важной, что считают не только главной современной тенденцией, но и закономерностью общественного развития. Однако чтобы увидеть истоки социализации, следует заглянуть вглубь истории – 19 век. Тогда на фоне бурного развития капитализма и все более явственного проявления как его успехов, так и «острых углов» (нарастающего социального неравенства, безработицы, бедности, незащищенности широких слоев населения и т.д.) все острее и громче становится критика капитализма.[6, с.23]
    \par
    В русле такой критики появилось и развивалось особое направление теории, названное «социальная экономика». И хотя оно разделилось на два течения, основные его взгляды можно свести к следующему:
    \begin{itemize}
        \item критическая характеристика капитализма свободной конкуренции с точки зрения его социальных последствий: социальное неравенство, эксплуатация, безработица, угнетение женщин и т.п.;
        \item критическое отношение к частной собственности: от частичного до полного ее отрицания;
        \item отказ от идеи непримиримости и антагонизма классов, признания возможности их сотрудничества;
        \item возможность решения социальных проблем связывалась не с классовой борьбой, а с воздействием на социально-экономические процессы;
        \item признание необходимости активной политики государства социально-экономической области как особой его миссии.
    \end{itemize}
    \par
    Основные функции социально-рыночной экономики можно представить следующим образом. Если учесть, что общей функцией любой экономической системы является обеспечение материальных условий жизнеспособности общества, то она, естественно, остается главной функцией и для социально-рыночной экономики. Тогда другими, специфическими ее функциями будут:
    \begin{itemize}
        \item функция создания условий для более широкого удовлетворения потребностей социально-рыночной экономики
        \item функция социальной защиты тех, кто оказался в силу ряда объективных причин вне рыночной «игры» (дети, безработные и т.п.);
        \item повышения благосостояния всех слоев общества на основе экономического развития страны;
    \end{itemize}

    \begin{center}
        \textbf{1.2 Главные компоненты СОР экономики}
    \end{center}
    \\
    \par
    Главными составляющими компонентами социальной рыночной экономики является:
    \begin{itemize}
        \item рыночное хозяйство.
        \item социальная сфера.
    \end{itemize}
    \par
    Их соотношение, структура не могут заранее обозначиться жестко и строго, поскольку будут реально зависеть от конкретных условий (от уровня развития страны, ее экономического потенциала, истории, менталитета и т.п.), поэтому могут быть неодинаковыми в разных странах.
    \par
    Такая вариативность структуры социальной рыночной экономики заложена самой этой формулой, в которой «социальная» и «рыночная» сосуществуют на равных. В этой двойственности есть и плюсы и минусы. Минус состоит в недостаточной определенности и четкости, а плюс в том, что на базе этой формулы возможны варианты сочетаний, составляющих компонент. Главное, чтобы они были в рамках основополагающих принципов данной системы, не выходили из ее координат.
    \par
    Когда говорится о социальной ориентации, чаще всего подразумеваются западноевропейские государства, которые, хотя и не добились таких высот в развитии экономики, но основной целью функционирования общества выдвигают обеспечение социального диалога, партнерства, высокого уровня и качества жизни, социальной защищенности и бесконфликтности. С особой трепетностью граждане нашего государства традиционно относились к шведской модели, в которой реализовались аспекты, до сих пор не воплощенные в реальности нашего государства.
    \par
    В основу количественных критериев социально-ориентированной экономики могут быть положены следующие: место социальной политики среди приоритетов развития, распределение социальных функций между государством и иными субъектами хозяйствования, доля государственной собственности в экономической системе, объем перераспределяемого ВВП. [1]
    \par
    Использование подобного рода параметров позволяет выделить четыре модели:
    \begin{itemize}
        \item либеральную
        \item консервативную
        \item социал-демократическую
        \item рудиментарную
    \end{itemize}
    Либеральный режим (Великобритания) основывается на приоритете рыночного механизма, относительно низком объеме ВВП, перераспределяемом через налоговую и бюджетную систему (не более 40%), осуществлении преимущественно пассивной политики на рынке труда, ориентированной на выплаты пособий по безработице, высоком удельном весе частных и общественных компаний в сфере производства частных услуг. Определенное государственное вмешательство в функционирование экономики несущественно изменяет условия жизни населения, которые обеспечивает рынок.
    \par
    Консервативный режим основан на рыночной логике распределения, относительно большой доле перераспределяемого ВВП (около 50%), ориентации на поддержание полной занятости, развитой системе социального диалога и партнерства. Существенное внимание уделяется проблемам оказания социальной помощи и социального обеспечения. Страховые фонды формируются в основном за счет работодателей. Подобный режим развивается в Германии, Франции, Италии, Австрии и Бельгии.
    \par
    Социал-демократический режим реализуется посредством перераспределения 50–60% производимого ВВП, активной политики на рынке труда, ориентации социальной политики на конкретного человека. Для экономики характерны высокий уровень дотаций и субсидий. Чрезмерные трансферты обеспечиваются высокими ставками налогообложения, что существенно снижает стимулы для предпринимательства и трудовой деятельности. В последнее время этот режим подвергается существенным модификациям. В качестве примеров можно определить Швецию, Данию, Норвегию и Финляндию.
    \par
    Рудиментарный режим характерен для наименее развитых стран региона, в которых высок уровень безработицы, объемы перераспределяемого ВВП могут существенно колебаться от 40% (в Испании) до 60% (в Греции). Социальная политика рассчитана на наиболее бедные категории граждан.
    \par
    Наибольший интерес для рассмотрения заслуживают шведская модель и современные попытки ее реформирования. Это обусловлено следующими моментами:
    \begin{itemize}
        \item Во-первых, здесь традиционно в качестве приоритетов социально-экономического развития рассматривают полную занятость и выравнивание доходов.
        \item Во-вторых, в данном государстве с 1932 г. (за исключением периода 1976–1982 гг.) у власти находятся социал-демократы. В-третьих, здесь
        \item В-третьих, представители данной нации очень близко воспринимают идею равенства.
    \end{itemize}

    \begin{center}
        \textbf{1.3 Черты и принципы СОР}
    \end{center}
    \\
    \par
    Основные черты модели социально-ориентированной рыночной экономики:
    \begin{itemize}
        \item обеспечение полной занятости населения;
        \item смешанная (государственно-частная) экономика. Одновременное наличие высокой доли государственной собственности и развитого института частной собственности;
        \item социальная безопасность, социальная справедливость и социальный прогресс (путём проведения государством мероприятий по перераспределению в форме оказания социальной помощи, социальных пенсий и уравнивающих платежей, субсидий, дотаций, прогрессивной шкалы подоходного налога и т. д., через систему социального обеспечения: пенсионное, медицинское страхование, страхование по безработице и по уходу, от несчастного случая, через трудовое и социальное законодательство);
        \item частная собственность на средства производства и свободное ценообразование;
        \item создание условий для конкуренции и обеспечение конкуренции (например, путём антимонопольного законодательства, законов против недобросовестной конкуренции);
        \item сознательная политика укрепления конъюнктуры экономического роста;
        \item политика стабильной валюты (в том числе через независимый эмиссионный банк);
        \item свобода внешней торговли, валютный обмен.
    \end{itemize}
    \par
    Модель социально-ориентированной рыночной экономики базируется на ряде основополагающих принципов:
    \begin{itemize}
        \item конституционные гарантии личных прав и свобод граждан;
        \item свобода предпринимательства и рыночной конкуренции;
        \item свободное ценообразование;
        \item возможность частной собственности на средства производства;
        \item равенство всех форм собственности (государственная, частная, акционерная, коллективная и т.д.);
        \item гарантии неприкосновенности частной собственности;
        \item независимый Центробанк, свободный валютный рынок, стабильная национальная валюта;
        \item низкий, близкий к нулю уровень инфляции;
        \item свободная внешняя торговля;
        \item низкие, близкие к нулю кредитные ставки;
        \item высокая ответственность бизнеса за своих работников;
        \item прямая зависимость благосостояния граждан от результатов их труда;
        \item высокий уровень социальной защиты нетрудоспособных и социально уязвимых категорий населения;
        \item социальное партнерство между государством, профсоюзами и союзами предпринимателей;
        \item низкий уровень коррупции;
        \item свободные выборы, сменяемость власти, независимая судебная система;
    \end{itemize}
    \par
    Таким образом, социально-ориентированная рыночная экономика, с одной стороны, смягчает такие присущие рыночной экономике признаки как высокий уровень социального неравенства,эксплуатация труда, высокая безработица, а с другой стороны, не подразумевает всеобщей уравниловки как плановая экономика, поскольку уровень благосостояния граждан напрямую зависит от результатов их труда.

    \begin{center}
        \textbf{1.4 Роль государства в СОР}
    \end{center}
    \\
    \par
    Государство в данной рыночной модели играет роль регулятора, распределителя,
    можно сказать, арбитра, обеспечивающего справедливое взаимодействие между бизнесом и
    населением. Оно законодательно создает условия для свободного развития рыночной экономики и,
    одновременно, выступает гарантом защищенности интересов граждан, в том числе, наиболее
    социально-уязвимых слоев общества: пенсионеров, инвалидов, и т.д.
    \par
    Его главными целями являются:
    \begin{itemize}
        \item рост благосостояния граждан;
        \item финансирование базовых социальных услуг и развития инфраструктуры;
        \item экономическая стабильность;
        \item высокий уровень занятости населения;
        \item максимально благоприятные условия для населения и бизнеса.
    \end{itemize}
    \par
    При этом государство не противопоставляет интересы населения и бизнеса, социальную
    справедливость и экономическую эффективность, а пытается обеспечить наиболее выгодное
    взаимодействие этих двух групп, максимальный уровень компромисса между ними.
    Обеспечение оптимального уровня социального равенства и справедливости в
    социально-ориентированной рыночной экономике является не только политическим принципом
    (как в плановой экономике), а и сильным факторов достижения максимальной экономической
    эффективности.
    \par

    \begin{center}
        \textbf{1.5 Недостатки СОР}
    \end{center}
    \\
    \par
    Всегда есть и обратная сторона - недостатки, давайте их тоже рассмотрим для объективности.
    \begin{itemize}
        \item Высокая социальная нагрузка на бюджет. Такая модель требует высокого
        уровня социальных расходов бюджета, что достигается за счет высокого уровня
        налогообложения бизнеса. В странах с социально-ориентированной рыночной экономикой
        всегда очень высокие налоги, что снижает эффективность ведения бизнеса и тормозит
        экономическое развитие.
        \item Раздутые штаты предприятий. За счет программ стимулирования занятости
        невозможно достичь оптимальных штатов сотрудников на предприятиях, зачастую они
        перенасыщены. А это, в свою очередь, тоже снижает производительность труда и
        рентабельность бизнеса.
        \item Расходы на поддержку курса национальной валюты.</strong> Данная экономическая
        модель предполагает обеспечение стабильного курса национальной валюты, а это требует
        огромных золотовалютных резервов для проведения валютных интервенций при необходимости.
        \item Безвозвратное расходование природных ресурсов. И, наконец, данная
        экономическая модель не способствует сохранению и восстановлению природных ресурсов.
        Проблема, конечно, достаточно глобальная, но и сюда ее тоже можно отнести.
    \end{itemize}
    \par
    Ввиду этих недостатков, страны, использующие данную экономическую модель, как
    правило, имеют более низкие темпы прироста ВВП, чем страны с чисто рыночной моделью
    экономики.

    \newpage
    \begin{center}
        \textbf{2 Опыт СОРЭ в других странах}
        \\
        \HRule\\[0.5cm]
        \textbf{2.1 Модели и направления СОРЭ}
    \end{center}
    \\
    \par
    Существуют несколько основных направлений (моделей) социально-ориентированной рыночной
    экономики, каждое из которых имеет свои характерные особенности.
    \begin{itemize}
        \item Континентальная (Германская) модель.
        Для континентальной модели характерны очень высокие объемы перераспределения ВВП через
        госбюджет (около 50 процентов), высокоразвитая система социального партнерства, высокая занятость
        населения.
        Примеры стран с континентальной моделью социально-ориентированной рыночной экономики:
        \begin{itemize}
            \item Германия;
            \item Австрия;
            \item Бельгия;
            \item Нидерланды;
            \item Швейцария.
        \end{itemize}
        \item Англосаксонская модель.
        Для англосаксонской модели характерны чуть меньшие объемы перераспределения ВВП через
        госбюджет (около 40 процентов) и пассивная политика занятости.
        Примеры стран с англосаксонской моделью социально-ориентированной рыночной экономики:
        \begin{itemize}
            \item Великобритания;
            \item Ирландия;
            \item Канада.
        \end{itemize}
        \item Средиземноморская модель.
        Средиземноморская модель может характеризоваться разными уровнями перераспределения ВВП через
        госбюджет (от 40 до 60 процентов), а социальная политика здесь охватывает только самые уязвимые слои
        населения и не носит всеобъемлющего характера.
        Примеры стран со средиземноморской моделью социально-ориентированной рыночной экономики:
        \begin{itemize}
            \item Греция;
            \item Испания;
            \item Италия.
        \end{itemize}
        \item Скандинавская модель.
        Скандинавская модель отличается самым высоким уровнем перераспределения ВВП через госбюджет
        (около 50 \&- 60 процентов) и самой активной социальной политикой. В странах, использующих эту
        экономическую модель, благосостояние народа провозглашается как высшая цель экономического
        развития государства.
        Примеры стран со скандинавской моделью социально-ориентированной рыночной экономики:
        \begin{itemize}
            \item Швеция;
            \item Дания;
            \item Норвегия;
            \item Финляндия.
        \end{itemize}
    \end{itemize}
    \par
    Наиболее социально-ориентированными среди перечисленных моделей являются скандинавская и
    континентальная, а в качестве примеров наиболее успешных стран с
    социально-ориентированной рыночной экономикой всегда приводят скандинавские страны (в
    первую очередь Швецию) и Германию. К слову, именно эти страны традиционно
    занимают ведущие места в рейтинге индекса счастья(показателя, характеризующего
    общий уровень удовлетворенности жизнью населения).

    \begin{center}
        \textbf{2.2 СОРЭ в Швеции}
    \end{center}
    \\
    \par
    Наибольший интерес для  рассмотрения заслуживают шведская модель и современные попытки ее реформирования.
    Это обусловлено следующими моментами.
    Во-первых, здесь традиционно в качестве приоритетов социально-экономического развития рассматривают полную занятость и выравнивание доходов.
    Во-вторых, в данном государстве с 1932 г. (за исключением периода 1976–1982 гг.) у власти находятся социал-демократы.
    В-третьих, здесь очень сильную позицию  имеют профсоюзы, которые оказывают существенное влияние на уровень и динамику заработной платы.
    В-четвертых, представители данной нации очень близко воспринимают идею равенства.
    \par
    Сам термин «шведская модель»  используется в нескольких контекстах:
    \begin{itemize}
        \item для определения типа экономической системы, в которой существенное влияние на развитие и динамику рыночной экономики оказывает государство;
        \item для характеристики специфической  ситуации на рынке труда, когда определенное место отводится системе проведения коллективных переговоров на основе активного участия профсоюзов;
        \item для определения активной политики на рынке труда и осуществления перераспределительной функции государства как приоритетов социально-экономического развития;
        \item как совокупность факторов социально-экономического и политического развития, которые обеспечивают достижение высокого уровня и качества
    \end{itemize}
    \par
    Основным элементом шведской социальной политики выступает социальное страхование.
    Его основная цель заключается в обеспечении граждан средствами в случае безработицы, болезни, необходимости получения медицинской помощи, рождения ребенка, по старости, в связи с несчастными случаями и травмами на производстве.
    Система страхования здоровья является средством создания условий для социально-экономического равенства.
    Она делает возможным получение медицинских услуг в случае необходимой неотложной медицинской помощи наравне с другими.
    Система социального страхования финансируется за счет налогов, взносов предпринимателей, трудящихся и неработающих по найму, доходов по процентам и вычетов из капитала различных фондов.
    Швеция явилась пионером во многих социальных начинаниях. Это касается, прежде всего, института социального партнерства, начало которому было положено в 1938 г., когда Шведская федерация профсоюзов и Шведская федерация работодателей подписали соглашение о мирном урегулировании трудовых конфликтов и необходимости заключения трудовых соглашений.
    Помимо этого, Швеция раньше всех пришла к необходимости и начала осуществлять активную политику на рынке труда, ввела запрет на строительство АЭС,
    стала перераспределять огромные средства через бюджет, а также выработала курс на построение общества всеобщего благоденствия.
    Несомненными достижениями шведского общества являются следующие:
    \begin{itemize}
        \item обеспечение высокого уровня жизни и социальных гарантий большей части населения общества без социальных потрясений и политических конфликтов;
        \item высокий уровень политической культуры, который позволил сформировать общественную систему диалога и кооперационный характер отношений между различными слоями населения;
        \item достижение высокого уровня социально-экономического развития и реализация таких важных экономических целей, как полная занятость, стабильный уровень цен, долговременный динамичный экономический рост;
        \item приоритет развития человеческого фактора, творческого начала в стимулировании трудовой деятельности, что нашло свое отражение в концепции «человеческого капитала». [2, с 154]
    \end{itemize}
    \par
    Анализ развития шведской экономики позволяет сделать вывод о том, что она построена в первую очередь на идеях кейнсианства относительно места и роли государства в экономической системе.
    Первая реформа экономики в этом государстве была проведена в условиях Великой депрессии 1930-х гг.
    Выход из создавшейся ситуации был найден за счет усиления государственного вмешательства в экономику, причем в Швеции с самого начала государство стало выполнять чрезмерные социальные функции.
    Основоположником шведской модели считается Г. Мюрдаль, который совершенно правильно обосновал связь развития техники и технологии с прогрессом социума, ибо все, что делается, целью своей имеет благо для человеческого общества.
    Стабильность в обществе была достигнута за счет компромисса между государством, предпринимателями и трудящимися, которые признали взаимные уступки друг другу. Рабочие отказались от проведения широкомасштабных политических акций, общенациональных забастовок и призывов к национализации собственности. Наниматели признали за государством право на осуществление социальных реформ. В результате сформировалась особая культура, в рамках которой все проблемы общества решались только мирным путем. Была достигнута, по сути, максимальная степень государственного вмешательства в рыночную систему. В период расцвета шведская модель характеризовалась разветвленной и комплексной системой социальной защиты населения.
    Более 50 процентов ВВП проходило через перераспределительные каналы, гарантировалось бесплатное образование, возможности получения бесплатных медицинских услуг, пособия по болезни или уходу за ребенком составляли 90 процентов заработка.
    Чрезмерные социальные выплаты тяжелым бременем легли на плечи предпринимателей и частного сектора экономики.
    Высокие ставки налогов перестали стимулировать инвестиционную активность, экономическая модель стала давать сбои.
    \par
    Кризис политической, экономической  и социальной действительности стал проявляться в снижении или даже отсутствии темпов экономического роста, развертывании инфляционных процессов, росте безработицы.
    Шведская экономика, функционировавшая в условиях полной занятости, столкнулась с невиданным ранее уровнем безработицы.
    Наибольшего значения в 9.9 процентов он достиг в 1997 г. Выход был найден за счет сокращения государственного сектора в экономике, проведения консервативных преобразований.
    Это происходило накануне присоединения к ЕС. Шведское законодательство было приведено в соответствие с принципами Евросоюза.
    Реально это выразилось в существенном снижении ставок налогообложения, что не замедлило позитивно сказаться на экономике; сокращении масштабов и размеров социальных выплат, результатом чего стало углубление дифференциации в обществе.
    Де-регулирование экономики позволило повысить конкурентоспособность шведских товаров на мировом рынке, уменьшить долю традиционных и повысить долю наукоемких производств; возросла степень концентрации капитала.
    Экономическая система стала постепенно переходить к фазе оживления деловой активности, вследствие чего стала снижаться безработица.
    На современном этапе уровень последней стабилизировался в пределах полной занятости.

    \begin{center}
        \textbf{2.3 СОРЭ в Германии}
    \end{center}
    \\
    \par
    Если сравнивать шведскую экономику, например, с немецкой, то можно отметить определенное сходство и отсутствие элементов особой оригинальности.
    Несомненно, в начальный момент развития шведской модели определенные ее черты были новы и неповторимы, но затем другие государства стали повторять некоторые из них.
    Сегодня можно констатировать, что процессы развития социальной ориентации стран континентальной Европы и Скандинавского полуострова идут навстречу друг другу.
    Первые идут по пути формирования справедливого, высокоразвитого социально-ориентированного общества на наднациональном уровне, а скандинавским государствам пришлось несколько либерализировать свои экономики с целью придания им относительной гибкости, способности принимать и быстро адаптироваться к переменам, уменьшению перегруженности государственного бюджета социальными расходами, но сохранению при этом высокого уровня социальных гарантий в обществе.
    \par
    Наиболее развитой страной  европейского континента является Германия, которая по объему ВВП уступает только США и Японии.
    Объем ВВП на душу населения составляет в стране 23836 долл. в год (2003г.) (по паритету покупательной способности) Экономическая система современной Германии носит наименование «социального рыночного хозяйства».
    Она характеризуется достаточно существенной ролью государства в экономике, особенно в сравнении с США или Великобританией.
    Основу системы составляет деятельность государства, которое пытается осуществить перераспределение социальных благ между всеми членами общества; при выполнении своих функций оно опирается на крупные банки.
    Позиции данных финансовых институтов в экономике Германии с учетом фактического влияния на государство и бизнес оказываются существенно сильнее, чем в других государствах.
    \par
    Впервые термин «социальное  рыночное хозяйство» и основные характеристики этого феномена были изложены в работе А. Мюллера-Армака «Регулирование экономики и рыночное хозяйство» 1947 г.
    Популярность новой теории, пытавшейся соединить  рыночную экономику и социальную направленность системы, была высочайшей.
    Дело в том, что мировой кризис 1929–1933 гг. показал неэффективность саморегулирования рыночной экономики.
    В этой обстановке в среде экономистов стали складываться два подхода, каждый из которых претендовал на главенствующее положение в объяснении дальнейшего развития рынка: кейнсианство и неолиберализм.
    Сторонники первого обосновывали необходимость государственного регулирования экономики; другие пытались переосмыслить и усовершенствовать взгляды неоклассической школы применительно к новым реалиям. [1]
    \par
    Современная немецкая модель характеризуется следующими чертами.
    Во-первых, важной чертой является индивидуальная свобода как условие функционирования рыночных механизмов и децентрализованного принятия решений.
    Это обеспечивается спецификой осуществляемых государством функций, среди которых следует выделить следующие:
    \begin{itemize}
        \item политика хозяйственного порядка, то есть обеспечение свободы деятельности хозяйствующих агентов на основе поддержания конкуренции и противодействия монополистической деятельности;
        \item поощрение малого и среднего бизнеса, развитие устойчивого и многочисленного среднего класса;
        \item социальная политика, направленная на защиту тех, кто по определенным причинам не может обеспечить себе достойное существование (инвалиды, пенсионеры, безработные), а также перераспределение доходов в обществе с целью устранения резких перекосов в распределении;
        \item экологическая политика, направленная на сохранение и поддержание окружающей среды;
        \item политика экономического роста и структурных сдвигов путем сглаживания конъюнктурных колебаний и ориентации на такие макроэкономические цели, как стабильность денежной единицы, полная занятость и равновесие платежного баланса.
    \end{itemize}
    \par
    Следующей чертой немецкой модели является защита и поощрение конкуренции.
    Чрезмерное внимание этому аспекту обусловлено тем, что Германия долгое время считалась «классической страной картелей»: их бурный рост в конце XIX – начале XX вв. приводил в смятение как либералов, считавших их угрозой индивидуальной свободе, так и социалистов, принимавших их за ростки нового общества.
    \par
    Третья черта рассматриваемой  модели – перераспределение собственности и доходов.
    Как известно, в рыночной экономике доход рассматривается как плата за использование производственных факторов – труда (в статистике Германии принято говорить о «доходах от несамостоятельных видов деятельности») и капитала («доходы от предпринимательской деятельности и имущества», то есть прибыль, процент, арендная плата).
    \par
    Еще одной чертой рассматриваемой немецкой модели социального рыночного хозяйства является соучастие работников в управлении предприятием.
    Закон об уставе предприятия 1972 г. ввел институт советов предприятий, а Закон о соучастии в управлении 1974 г. обязал крупные предприятия до половины мест в наблюдательных советах предоставлять наемным работникам и их представителям.
    Таким образом, модель социального рыночного хозяйства получила важный момент стабилизации отношений в обществе, поскольку таким образом можно снимать противоречия между собственниками, управляющими и наемными работниками.
    \par
    Итак, социально ориентированная рыночная экономика-это высокоэффективная открытая экономика с развитым предпринимательством и рыночной инфраструктурой, действенным государственным регулированием доходов, заинтересовывающем предпринимателей в расширении и совершенствовании производства, а наемных работников в высокопроизводительном труде.
    Она гарантирует высокий уровень благосостояния добросовестно работающим членам общества, достойное социальное обеспечение нетрудоспособным (престарелым, инвалидам, женщинам, находящимся в отпуске по уходу за ребенком); эффективную охрану жизни, здоровья, прав и свобод всем гражданам.
    \par
    Белорусская модель экономики схожа с немецкой и шведской по следующим принципам: социальное равенство, поощрение конкуренции, полная занятость, социальная политика, поощрение предпринимательства, профсоюзная политика, стимулирование трудовой деятельности.

    \newpage
    \begin{center}
        \textbf{3 Направления СОРЭ в РБ}
    \end{center}

    \newpage
    \begin{center}
        \textbf{3.1 Социально-экономическое развитие как направление СОРЭ в РБ}
    \end{center}

    \newpage
    \begin{center}
        \textbf{3.2 Основные проблемы социально-экономического развития РБ}
    \end{center}

    \newpage
    \begin{center}
        \textbf{\LARGE{ЗАКЛЮЧЕНИЕ}}
    \end{center}

    \newpage
    \begin{center}
        \renewcommand\refname{СПИСОК ИСПОЛЬЗОВАННЫХ ИСТОЧНИКОВ}
        \begin{thebibliography}{9}
            \bibitem{https://president.gov.by} https://president.gov.by/ru/events/utverzhdeny-vazhneyshie-parametry-prognoza-socialno-ekonomicheskogo-razvitiya-belarusi-na-2022-god.
            \bibitem{etalone.by/25} https://etalonline.by/novosti/korotko-o-vazhnom/sotsialno-ekonomicheskoe-razvitie-do-2025-goda/
            \bibitem{etalone.by/22} https://etalonline.by/novosti/korotko-o-vazhnom/sotsialno-ekonomicheskogo-razvitiya-respubliki-belarus/
            \bibitem{wiki1} Социальное рыночное хозяйство: Теория и этика экономического порядка в России и Германии. / Пер. с нем. под ред. В. С. Автономова. — СПб. : Экономическая школа, 1999. — 367 с. — (Этическая экономия: исследования по этике, культуре и философии хозяйства; Вып.6). — ISBN 5-900428-43-5
            \bibitem{wiki2} Социальное рыночное хозяйство / С. И. Невский, А. А. Ткаченко // Большая российская энциклопедия : [в 35 т.] / гл. ред. Ю. С. Осипов. — М. : Большая российская энциклопедия, 2004—2017.
            \bibitem{wiki3} Давыдова Т. Е. Формирование и историческое развитие концепции социального рыночного хозяйства // Историко-экономические исследования. 2006. № 1.
            \bibitem{wiki4} Социальное рыночное хозяйство: концепция, практический опыт и перспективы применения в России / Под ред. Р. М. Нуреева. — М.: Издательский дом ГУ-ВШЭ, 2007.
            \bibitem{wiki5} Социальное рыночное хозяйство — основоположники и классики : сборник научных трудов / авт. предисл. К. Кроуфорд; ред.-сост. К. Кроуфорд, С. И. Невский, Е. В. Романова и др. — М. : Весь Мир, 2017. — 418 с. : ил. — ISBN 978-5-7777-0676-8
            \bibitem{wiki6} История концепции социального рыночного хозяйства в Германии / Ред.-сост.: С. И. Невский, А. Г. Худокормов. — М.: ИНФРА-М, 2022. — 212 с. ISBN 978-5-16-017090-9
            \bibitem{indexHap} Индекс счастья - https://fingeniy.com/indeks-schastya/
        \end{thebibliography}
    \end{center}
\end{document}