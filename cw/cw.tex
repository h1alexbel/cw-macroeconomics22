% #MIT License
% #
% #Copyright (c) 2022 Aliaksei Bialiauski
% #
% #Permission is hereby granted, free of charge, to any person obtaining a copy
% #of this software and associated documentation files (the "Software"), to deal
% #in the Software without restriction, including without limitation the rights
% #to use, copy, modify, merge, publish, distribute, sublicense, and/or sell
% #copies of the Software, and to permit persons to whom the Software is
% #furnished to do so, subject to the following conditions:
% #
% #The above copyright notice and this permission notice shall be included in all
% #copies or substantial portions of the Software.
% #
% #THE SOFTWARE IS PROVIDED "AS IS", WITHOUT WARRANTY OF ANY KIND, EXPRESS OR
% #IMPLIED, INCLUDING BUT NOT LIMITED TO THE WARRANTIES OF MERCHANTABILITY,
% #FITNESS FOR A PARTICULAR PURPOSE AND NONINFRINGEMENT. IN NO EVENT SHALL THE
% #AUTHORS OR COPYRIGHT HOLDERS BE LIABLE FOR ANY CLAIM, DAMAGES OR OTHER
% #LIABILITY, WHETHER IN AN ACTION OF CONTRACT, TORT OR OTHERWISE, ARISING FROM,
% #OUT OF OR IN CONNECTION WITH THE SOFTWARE OR THE USE OR OTHER DEALINGS IN THE
% #SOFTWARE.

\documentclass[14pt,a4paper]{article}

\usepackage[a4paper, left={1.35in}, outer={0.55in}, bottom={1in}]{geometry}
\usepackage[utf8]{inputenc}
\usepackage[T1]{fontenc}
\usepackage{amsmath}
\usepackage[russian]{babel}
\usepackage{lipsum}

\addto\captionsrussian{
    \renewcommand{\contentsname}%
    {СОДЕРЖАНИЕ}%
}

\newenvironment{linez}
{\trivlist\nopagebreak
\parindent0pt
\item\relax\obeylines}
{\par
\vspace{3pt}%
\endtrivlist}

\begin{document}
    \begin{titlepage}
        \newcommand{\HRule}{\rule{\linewidth}{0mm}}
        \center
        \textsc{\Large МИНИСТЕРСТВО ОБРАЗОВАНИЯ РЕСПУБЛИКИ БЕЛАРУСЬ}\\[0cm]
        \renewcommand{\baselinestretch}{1.5}
        \textsc{\Large УО «БЕЛОРУССКИЙ ГОСУДАРСТВЕННЫЙ ЭКОНОМИЧЕСКИЙ УНИВЕРСИТЕТ»}\\[2cm]
        \textsc{ Кафедра экономической политики}\\[0.5cm]
        \HRule\\[1cm]
        {\bfseries КУРСОВАЯ РАБОТА}\\[0.4cm]
        \HRule\\[0cm]
        \begin{linez}
            \\ по дисциплине: {\bfseries Макроэкономика}
            \\ на тему: {\bfseries Социально-ориентированная рыночная экономика: черты, принципы и направления. Особенности белорусской модели развития.}
        \end{linez}
        \HRule\\[2cm]

        \begin{minipage}{0.4\textwidth}
            \begin{flushleft}
                \begin{tabular*}{\textwidth}{@{}p{.45\textwidth}@{\extracolsep{\fill}}r@{}}
                    \raggedright
                    \large{Студент}
                    \vspace{-5pt}
                    \\ФЭМ, 2-й курс, 21-ДКС
                \end{tabular*}
            \end{flushleft}
        \end{minipage}
        ~
        \begin{minipage}{0.4\textwidth}
            \begin{flushleft}
                \hspace{1.5cm}
                А.В. \textsc{Белявский}
            \end{flushleft}
        \end{minipage}
        \HRule\\[0.5cm]
        \begin{minipage}{0.4\textwidth}
            \begin{flushleft}
                \large{Руководитель}\\
                \vspace{-5pt}
                кандидат экон.наук, доцент
            \end{flushleft}
        \end{minipage}
        ~
        \begin{minipage}{0.4\textwidth}
            \begin{flushleft}
                \hspace{1.5cm}
                A.Д. \textsc{Якутович}
            \end{flushleft}
        \end{minipage}

        \HRule\\[5.5cm]
        {\large МИНСК 2022}
        \vfill
    \end{titlepage}

    \newpage
    \begin{center}
        \textbf{\Large{РЕФЕРАТ}}
    \end{center}
    \par
    Курсовая работа: ?с., ? табл., ? источник.
    \\
    \par
    \large{СЛОВА}
    \\
    \par
    \textbf{Объект исследования - } Социально-ориентированная рыночная экономика.
    \par
    \textbf{Предмет исследования - } модель социально-экономического развития Республики Беларусь.
    \par
    \textbf{Цель работы:} выявить условия формирования социально-ориентированной экономики, проанализировать модель социально-ориентированной рыночной экономики в Республике Беларусь, определить проблемы развития социально-ориентированной экономики республики и соответствующих перспектив.
    \par
    \textbf{Методы исследования:} метод описания, сравнительного анализа, системного подхода, исторический метод.
    \par
    \textbf{Исследования и разработки:} проведен анализ модели социально-ориентированной экономики в Республике Беларусь, в результате которого рассмотрены особенности ее реализации и пути совершенствования в Республике Беларусь
    \par
    \textbf{Элементы научной новизны:} определены и сформулированы основные проблемы модели социально-ориентированной экономики в Республике Беларусь и предложены способы их решения.
    \par
    \textbf{Область возможного практического применения:} результаты, полученные в курсовой работе, могут быть использованы на лекционных и семинарских занятиях по изучению курса «макроэкономика».
    \\
    \\
    \tabcolsep365pt
    \hfill\begin{tabular}{lp{.5\linewidth}@{}}
              \_\_\_\_\_\_\_\_\_\_
              \\
              (подпись студента)\\
    \end{tabular}

    \newpage
    \begin{center}
        \textbf{\LARGE{ESSAY}}
    \end{center}
    \par
    Course work: ?p., ? tab., ? sources.
    \\
    \par
    \large{WORDS}
    \\
    \par
    \textbf{The object of study} is a socially-oriented market economy.
    \par
    \textbf{The subject of study} is a model of socio-economic development of the Republic of Belarus.
    \par
    \textbf{Purpose of the work:} to identify the conditions for the formation of a socially oriented economy, analyze the model of a socially oriented market economy in the Republic of Belarus, identify the problems of development of a socially oriented economy of the republic and the corresponding prospects.
    \par
    \textbf{Research methods:} method of description, comparative analysis, systematic approach, historical method.
    \par
    \textbf{Research and development:} an analysis of the model of a socially oriented economy in the Republic of Belarus was carried out, as a result of which the features of its implementation and ways of improvement in the Republic of Belarus were considered.
    \par
    \textbf{Elements of scientific novelty:} the main problems of the model of a socially oriented economy in the Republic of Belarus are identified and formulated, and ways to solve them are proposed.
    \par
    \textbf{Area of possible practical application:} the results obtained in the course work can be used in lectures and seminars on the study of the course ``macroeconomics''.
    \\
    \\
    \tabcolsep365pt
    \hfill\begin{tabular}{lp{.5\linewidth}@{}}
              \_\_\_\_\_\_\_\_\_\_\_\_\_\_
              \\
              (signature of the student)\\
    \end{tabular}

    \newpage
    \begin{center}
        \tableofcontents
        \addcontentsline{toc}{section}{РЕФЕРАТ}
        \addcontentsline{toc}{section}{ВВЕДЕНИЕ}
        \addcontentsline{toc}{section}{1 Социально-ориентированная рыночная(СОР) экономика}
        \addcontentsline{toc}{section}{1.1 Главные компоненты СОР экономики}
        \addcontentsline{toc}{section}{1.2 Черты и принципы СОР}
        \addcontentsline{toc}{section}{2 Опыт и направления СОР в других странах}
        \addcontentsline{toc}{section}{2.1 СОР в Швеции}
        \addcontentsline{toc}{section}{2.2 СОР в Германии}
        \addcontentsline{toc}{section}{2.3 СОР в Великобритании}
        \addcontentsline{toc}{section}{3 Направления СОР в РБ}
        \addcontentsline{toc}{section}{3.1 Социально-экономическое развитие как направление СОР в РБ}
        \addcontentsline{toc}{section}{3.2 Основные проблемы социально-экономического развития РБ}
        \addcontentsline{toc}{section}{ЗАКЛЮЧЕНИЕ}
        \addcontentsline{toc}{section}{СПИСОК ИСПОЛЬЗОВАННЫХ ИСТОЧНИКОВ}
    \end{center}

    \newpage
    \begin{center}
        \textbf{\LARGE{ВВЕДЕНИЕ}}
    \end{center}
    \\
    \par
    Девяностые годы XX века были судьбоносными для экономики Беларуси. Распался Советский Союз, рухнули прежние принципы организации и функционирования экономики, практически все бывшие союзные республики оказались в глубоком политическом и экономическом кризисе.
    \par
    Именно становление экономической системы рыночного типа выдвигает в число первоочередных такие вопросы, как преобразование механизма управления, изменение принципов мотивации труда. Актуальность этих вопросов  для Республики Беларусь вызвана сложностью переходного периода, выбором пути социально-экономического развития, кризисом всех сфер общества, необходимостью создания собственной модели экономики.
    \par
    Возникла необходимость в сжатые сроки создать собственную государственность, национальную банковскую систему, ввести национальную валюту, искоренить гиперинфляцию, насытить рынок элементарными потребительскими товарами, поддерживать отечественных производителей, сдерживать рост безработицы, защитить социально наиболее уязвимые слои населения, определить перспективную модель социально-экономического развития страны, овладеть навыками конкурирования на мировых рынках товаров и услуг.
    \par
    Актуальность темы исследования  курсовой работы связана с необходимостью определения целей и условий  формирования социально - ориентированной рыночной экономики, опираясь на опыт других стран, в определении проблем и перспектив социально-экономического развития Республики Беларусь. Анализ этих вопросов даст не только четкую оценку идеям, использовавшимся при формировании хозяйства нашей Республики, но и поможет определить и сформулировать новые, наиболее созвучные времени и мировым тенденциям развития экономики государства.
    \par
    Главной целью курсовой работы является необходимость понять условия  формирования социально-ориентированной экономики, рассмотреть модель социально-ориентированной рыночной экономики в нашей стране, определить  проблемы ее развития и соответствующие перспективы.
    \par
    Задачи курсовой работы:
    \begin{itemize}
        \item выявить понятие и сущность социально-ориентированной экономики;
        \item рассмотреть основные черты и модели социально-ориентированной экономики;
        \item проанализировать модель экономического развития Республики Беларусь и ее дальнейшие перспективы.
    \end{itemize}
    \par
    В ходе написания курсовой работы использовались учебные пособия, нормативные документы, статистические данные, интернет источники.

    \newpage
    \begin{center}
        \textbf{1 Социально-ориентированная рыночная(СОР) экономика}
    \end{center}
    \\
    \par
    Социа́льно-ориенти́рованная ры́ночная эконо́мика (социа́льное ры́ночное хозя́йство) (нем. Soziale Marktwirtschaft) — экономическая система, организованная на основе рыночной саморегуляции, при которой координация действий осуществляется на основе взаимодействия на рынках свободных частных производителей и свободных индивидуальных потребителей. Модель социально-рыночной экономики исходит из требования, что ни государство, ни частный бизнес не вправе иметь полный контроль над экономикой, а должны служить людям. В этой разновидности смешанной экономики, так же как и в рыночной экономике, структура распределения ресурсов определяется исключительно решениями самих потребителей, поставщиков ресурсов и частных фирм. Однако при этом экономически более сильные обязаны поддерживать более слабых. Роль государства заключается в развитии чувства взаимной ответственности всех участников на рынке и в корректировке несправедливых тенденций в конкуренции, торговле и распределении доходов. Система рассматривалась как альтернатива «laissez-faire» капитализму и социализму.
    \par
    Концепцию начал реализовывать Людвиг Эрхард, министр экономики, а впоследствии федеральный канцлер Германии. Название системы социально-рыночной экономики дал в 1947 году экономист Альфред Мюллер-Армак. Важный вклад в развитие также внесли Франц Бёмruen, Вальтер Ойкен, Франц Оппенгеймер, Вильгельм Рёпке, Александер Рюстов, Жак Фреско, Концепция стала одним из существенных элементов идеологии христианско-демократического движения и получила поддержку со стороны социальных консерваторов, социальных либералов и социал-демократов.
    \par
    Термин «социально-ориентированная экономика» является в последнее время достаточно употребляемым, хотя в обыденном токовании представление о нем достаточно расплывчато. Не совсем понято, является ли ориентация на высокие социальные параметры характеристикой всех экономик, или она присуща только некоторым разновидностям. Как известно, любая экономическая система имеет своей целью достижение высокого уровня удовлетворения материальных и духовных потребностей людей, а потому обладает социальными моментами. Действительно, не может же экономика развиваться ради себя самой: конечным результатом и целью жизнедеятельности является человек. Тем не менее, под социально-ориентированной понимается такой тип экономической системы, который отвечает определенным критериям. Можно толковать социально-ориентированную экономику как систему, в которой наиболее значимой является перераспределительная функция государства, позволяющая устранить существенную дифференциацию доходов в обществе и ликвидировать полюса бедности и богатства. Данное определение будет не совсем точным и правильным, поскольку любое государство занимается перераспределением доходов посредством функционирования бюджетной системы, поэтому вряд ли можно говорить о социальности каждой экономики.
    \par
    Социализация экономики  стала реальностью с 20 века. Многие исследователи признают ее настолько  важной, что считают не только главной  современной тенденцией, но и закономерностью общественного развития. Однако чтобы увидеть истоки социализации, следует заглянуть вглубь истории – 19 век. Тогда на фоне бурного развития капитализма и все более явственного проявления как его успехов, так и «острых углов» (нарастающего социального неравенства, безработицы, бедности, незащищенности широких слоев населения и т.д.) все острее и громче становится критика капитализма.[6, с.23]
    \par
    В русле такой критики  появилось и развивалось особое направление теории, названное «социальная  экономика». И хотя оно разделилось  на два течения, основные его взгляды  можно свести к следующему:
    \begin{itemize}
        \item критическая характеристика капитализма свободной конкуренции с точки зрения его социальных последствий: социальное неравенство, эксплуатация, безработица, угнетение женщин и т.п.;
        \item критическое отношение к частной собственности: от частичного до полного ее отрицания;
        \item отказ от идеи непримиримости и антагонизма классов, признания возможности их сотрудничества;
        \item возможность решения социальных проблем связывалась не с классовой борьбой, а с воздействием на социально-экономические процессы;
        \item признание необходимости активной политики государства социально-экономической области как особой его миссии.
    \end{itemize}
    \par
    Основные функции социально-рыночной экономики можно представить следующим образом. Если учесть, что общей функцией любой экономической системы является обеспечение материальных условий жизнеспособности общества, то она, естественно, остается главной функцией и для социально-рыночной экономики. Тогда другими, специфическими ее функциями будут:
    \begin{itemize}
        \item функция создания условий для более широкого удовлетворения потребностей социально-рыночной экономики
        \item функция социальной защиты тех, кто оказался в силу ряда объективных причин вне рыночной «игры» (дети, безработные и т.п.);
        \item повышения благосостояния всех слоев общества на основе экономического развития страны;
    \end{itemize}

    \newpage
    \begin{center}
        \textbf{1.1 Главные компоненты СОР экономики}
    \end{center}
    \\
    \par
    Главными составляющими  компонентами социальной рыночной экономики  является:
    \begin{itemize}
        \item рыночное хозяйство.
        \item социальная сфера.
    \end{itemize}
    \par
    Их соотношение, структура  не могут заранее обозначиться жестко и строго, поскольку будут реально  зависеть от конкретных условий (от уровня развития страны, ее экономического потенциала, истории, менталитета и т.п.), поэтому  могут быть неодинаковыми в разных странах.
    \par
    Такая вариативность структуры  социальной рыночной экономики заложена самой этой формулой, в которой  «социальная» и «рыночная» сосуществуют на равных. В этой двойственности есть и плюсы и минусы. Минус состоит  в недостаточной определенности и четкости, а плюс в том, что  на базе этой формулы возможны варианты сочетаний, составляющих компонент. Главное, чтобы они были в рамках основополагающих принципов данной системы, не выходили из ее координат.
    \par
    Когда говорится о социальной ориентации, чаще всего подразумеваются западноевропейские государства, которые, хотя и не добились таких высот в развитии экономики, но основной целью функционирования общества выдвигают обеспечение социального диалога, партнерства, высокого уровня и качества жизни, социальной защищенности и бесконфликтности. С особой трепетностью граждане нашего государства традиционно относились к шведской модели, в которой реализовались аспекты, до сих пор не воплощенные в реальности нашего государства.
    \par
    В основу количественных критериев социально-ориентированной экономики могут быть положены следующие: место социальной политики среди приоритетов развития, распределение социальных функций между государством и иными субъектами хозяйствования, доля государственной собственности в экономической системе, объем перераспределяемого ВВП. [1]
    \par
    Использование подобного  рода параметров позволяет выделить четыре модели:
    \begin{itemize}
        \item либеральную
        \item консервативную
        \item социал-демократическую
        \item рудиментарную
    \end{itemize}
    Либеральный режим (Великобритания) основывается на приоритете рыночного механизма, относительно низком объеме ВВП, перераспределяемом через налоговую и бюджетную систему (не более 40%), осуществлении преимущественно пассивной политики на рынке труда, ориентированной на выплаты пособий по безработице, высоком удельном весе частных и общественных компаний в сфере производства частных услуг. Определенное государственное вмешательство в функционирование экономики несущественно изменяет условия жизни населения, которые обеспечивает рынок.
    \par
    Консервативный  режим основан на рыночной логике распределения, относительно большой доле перераспределяемого ВВП (около 50%), ориентации на поддержание полной занятости, развитой системе социального диалога и партнерства. Существенное внимание уделяется проблемам оказания социальной помощи и социального обеспечения. Страховые фонды формируются в основном за счет работодателей. Подобный режим развивается в Германии, Франции, Италии, Австрии и Бельгии.
    \par
    Социал-демократический режим реализуется посредством перераспределения 50–60% производимого ВВП, активной политики на рынке труда, ориентации социальной политики на конкретного человека. Для экономики характерны высокий уровень дотаций и субсидий. Чрезмерные трансферты обеспечиваются высокими ставками налогообложения, что существенно снижает стимулы для предпринимательства и трудовой деятельности. В последнее время этот режим подвергается существенным модификациям. В качестве примеров можно определить Швецию, Данию, Норвегию и Финляндию.
    \par
    Рудиментарный режим характерен для наименее развитых стран региона, в которых высок уровень  безработицы, объемы перераспределяемого ВВП могут существенно колебаться от 40% (в Испании) до 60% (в Греции). Социальная политика рассчитана на наиболее бедные категории граждан.
    \par
    Наибольший интерес для  рассмотрения заслуживают шведская модель и современные попытки ее реформирования. Это обусловлено следующими моментами:
    \begin{itemize}
        \item Во-первых, здесь традиционно в качестве приоритетов социально-экономического развития рассматривают полную занятость и выравнивание доходов.
        \item Во-вторых, в данном государстве с 1932 г. (за исключением периода 1976–1982 гг.) у власти находятся социал-демократы. В-третьих, здесь
        \item В-третьих, представители данной нации очень близко воспринимают идею равенства.
    \end{itemize}

    \newpage
    \begin{center}
        \textbf{1.2 Черты и принципы СОР}
    \end{center}

    \newpage
    \begin{center}
        \textbf{2 Опыт и направления СОР в других странах}
    \end{center}

    \newpage
    \begin{center}
        \textbf{2.1 СОР в Швеции}
    \end{center}

    \newpage
    \begin{center}
        \textbf{2.2 СОР в Германии}
    \end{center}

    \newpage
    \begin{center}
        \textbf{2.3 СОР в Великобритании}
    \end{center}

    \newpage
    \begin{center}
        \textbf{3 Направления СОР в РБ}
    \end{center}

    \newpage
    \begin{center}
        \textbf{3.1 Социально-экономическое развитие как направление СОР в РБ}
    \end{center}

    \newpage
    \begin{center}
        \textbf{3.2 Основные проблемы социально-экономического развития РБ}
    \end{center}

    \newpage
    \begin{center}
        \textbf{\LARGE{ЗАКЛЮЧЕНИЕ}}
    \end{center}

    \newpage
    \begin{center}
        \renewcommand\refname{СПИСОК ИСПОЛЬЗОВАННЫХ ИСТОЧНИКОВ}
        \begin{thebibliography}{7}
            \bibitem{https://president.gov.by} https://president.gov.by/ru/events/utverzhdeny-vazhneyshie-parametry-prognoza-socialno-ekonomicheskogo-razvitiya-belarusi-na-2022-god.
            \bibitem{etalone.by/25} https://etalonline.by/novosti/korotko-o-vazhnom/sotsialno-ekonomicheskoe-razvitie-do-2025-goda/
            \bibitem{etalone.by/22} https://etalonline.by/novosti/korotko-o-vazhnom/sotsialno-ekonomicheskogo-razvitiya-respubliki-belarus/
        \end{thebibliography}
    \end{center}
\end{document}