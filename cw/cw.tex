% #MIT License
% #
% #Copyright (c) 2022 Aliaksei Bialiauski
% #
% #Permission is hereby granted, free of charge, to any person obtaining a copy
% #of this software and associated documentation files (the "Software"), to deal
% #in the Software without restriction, including without limitation the rights
% #to use, copy, modify, merge, publish, distribute, sublicense, and/or sell
% #copies of the Software, and to permit persons to whom the Software is
% #furnished to do so, subject to the following conditions:
% #
% #The above copyright notice and this permission notice shall be included in all
% #copies or substantial portions of the Software.
% #
% #THE SOFTWARE IS PROVIDED "AS IS", WITHOUT WARRANTY OF ANY KIND, EXPRESS OR
% #IMPLIED, INCLUDING BUT NOT LIMITED TO THE WARRANTIES OF MERCHANTABILITY,
% #FITNESS FOR A PARTICULAR PURPOSE AND NONINFRINGEMENT. IN NO EVENT SHALL THE
% #AUTHORS OR COPYRIGHT HOLDERS BE LIABLE FOR ANY CLAIM, DAMAGES OR OTHER
% #LIABILITY, WHETHER IN AN ACTION OF CONTRACT, TORT OR OTHERWISE, ARISING FROM,
% #OUT OF OR IN CONNECTION WITH THE SOFTWARE OR THE USE OR OTHER DEALINGS IN THE
% #SOFTWARE.

\documentclass[14pt,a4paper]{article}

\usepackage[a4paper, left={1.35in}, outer={0.55in}, top={1in}, bottom={1in}]{geometry}
\usepackage[utf8]{inputenc}
\usepackage[T1]{fontenc}
\usepackage{amsmath}
\usepackage[russian]{babel}
\usepackage{lipsum}

\addto\captionsrussian{
    \renewcommand{\contentsname}%
    {СОДЕРЖАНИЕ}%
}

\newenvironment{linez}
{\trivlist\nopagebreak
\parindent0pt
\item\relax\obeylines}
{\par
\vspace{3pt}%
\endtrivlist}

\begin{document}
    \begin{titlepage}
        \newcommand{\HRule}{\rule{\linewidth}{0mm}}
        \center
        \textsc{\Large МИНИСТЕРСТВО ОБРАЗОВАНИЯ РЕСПУБЛИКИ БЕЛАРУСЬ}\\[0cm]
        \renewcommand{\baselinestretch}{1.5}
        \textsc{\Large УО «БЕЛОРУССКИЙ ГОСУДАРСТВЕННЫЙ ЭКОНОМИЧЕСКИЙ УНИВЕРСИТЕТ»}\\[2cm]
        \textsc{ Кафедра экономической политики}\\[0.5cm]
        \HRule\\[1cm]
        {\bfseries КУРСОВАЯ РАБОТА}\\[0.4cm]
        \HRule\\[0cm]
        \begin{linez}
            \\ по дисциплине: {\bfseries Макроэкономика}
            \\ на тему: {\bfseries Социально-ориентированная рыночная экономика: черты, принципы и направления. Особенности белорусской модели развития.}
        \end{linez}
        \HRule\\[2cm]

        \begin{minipage}{0.4\textwidth}
            \begin{flushleft}
                \large{Студент}\\
                \vspace{-5pt}
                ФЭМ, 2-й курс, 21-ДКС
            \end{flushleft}
        \end{minipage}
        ~
        \begin{minipage}{0.4\textwidth}
            \begin{flushleft}
                \hspace{1.5cm}
                А.В. \textsc{Белявский}
            \end{flushleft}
        \end{minipage}
        \HRule\\[0.5cm]
        \begin{minipage}{0.4\textwidth}
            \begin{flushleft}
                \large{Руководитель}\\
                \vspace{-5pt}
                кандидат экон.наук, доцент
            \end{flushleft}
        \end{minipage}
        ~
        \begin{minipage}{0.4\textwidth}
            \begin{flushleft}
                \hspace{1.5cm}
                A.Д. \textsc{Якутович}
            \end{flushleft}
        \end{minipage}

        \HRule\\[5.5cm]
        {\large МИНСК 2022}
        \vfill
    \end{titlepage}

    \newpage
    \begin{center}
        \textbf{\Large{РЕФЕРАТ}}
    \end{center}
    \par
    Курсовая работа: 36с., 3 табл., 20 источников.
    \\
    \par
    \large{СОЦИАЛЬНО-ОРИЕНТИРОВАННАЯ ЭКОНОМИКА, ЛИБЕРАЛЬНЫЙ РЕЖИМ, КОНСЕРВАТИВНЫЙ РЕЖИМ, СОЦИАЛ-ДЕМОКРАТИЧЕСКИЙ РЕЖИМ,}
    \\
    \large{СОЦИАЛЬНО-ЭКОНОМИЧЕСКОЕ РАЗВИТИЕ, РУДИМЕНТАРНЫЙ РЕЖИМ}
    \\
    \par
    \textbf{Объект исследования - } Социально-ориентированная рыночная экономика.
    \par
    \textbf{Предмет исследования - } модель социально-экономического развития Республики Беларусь.
    \par
    \textbf{Цель работы:} выявить условия формирования социально-ориентированной экономики, проанализировать модель социально-ориентированной рыночной экономики в Республике Беларусь, определить проблемы развития социально-ориентированной экономики республики и соответствующих перспектив.
    \par
    \textbf{Методы исследования:} метод описания, сравнительного анализа, системного подхода, исторический метод.
    \par
    \textbf{Исследования и разработки:} проведен анализ модели социально-ориентированной экономики в Республике Беларусь, в результате которого рассмотрены особенности ее реализации и пути совершенствования в Республике Беларусь
    \par
    \textbf{Элементы научной новизны:} определены и сформулированы основные проблемы модели социально-ориентированной экономики в Республике Беларусь и предложены способы их решения.
    \par
    \textbf{Область возможного практического применения:} результаты, полученные в курсовой работе, могут быть использованы на лекционных и семинарских занятиях по изучению курса «макроэкономика».
    \\
    \\
    \tabcolsep365pt
    \hfill\begin{tabular}{lp{.5\linewidth}@{}}
              \_\_\_\_\_\_\_\_\_\_
              \\
              (подпись студента) \\
    \end{tabular}

    \newpage
    \begin{center}
        \textbf{\LARGE{ESSAY}}
    \end{center}
    \par
    Course work: 36p., 3 tab., 20 sources.
    \\
    \par
    \large{SOCIO-ORIENTED ECONOMY, LIBERAL MODEL, CONSERVATIVE MODEL,}
    \\
    \large{SOCIAL-DEMOCRATIC MODEL, SOCIO-ECONOMIC DEVELOPMENT,}
    \\
    \large{RUDIMENTARY MODEL}
    \\
    \par
    \textbf{The object of study} is a socially-oriented market economy.
    \par
    \textbf{The subject of study} is a model of socio-economic development of the Republic of Belarus.
    \par
    \textbf{Purpose of the work:} to identify the conditions for the formation of a socially oriented economy, analyze the model of a socially oriented market economy in the Republic of Belarus, identify the problems of development of a socially oriented economy of the republic and the corresponding prospects.
    \par
    \textbf{Research methods:} method of description, comparative analysis, systematic approach, historical method.
    \par
    \textbf{Research and development:} an analysis of the model of a socially oriented economy in the Republic of Belarus was carried out, as a result of which the features of its implementation and ways of improvement in the Republic of Belarus were considered.
    \par
    \textbf{Elements of scientific novelty:} the main problems of the model of a socially oriented economy in the Republic of Belarus are identified and formulated, and ways to solve them are proposed.
    \par
    \textbf{Area of possible practical application:} the results obtained in the course work can be used in lectures and seminars on the study of the course ``macroeconomics''.
    \\
    \\
    \tabcolsep365pt
    \hfill\begin{tabular}{lp{.5\linewidth}@{}}
              \_\_\_\_\_\_\_\_\_\_\_\_\_\_
              \\
              (signature of the student) \\
    \end{tabular}

    \newpage
    \begin{center}
        \fontsize{10}{}
        \tableofcontents
        \addcontentsline{toc}{section}{РЕФЕРАТ}
        \addcontentsline{toc}{section}{ВВЕДЕНИЕ}
        \addcontentsline{toc}{section}{1 Социально-ориентированная рыночная экономика(СОРЭ)}
        \addcontentsline{toc}{section}{1.1 Определение и ключевые понятия}
        \addcontentsline{toc}{section}{1.2 Главные компоненты СОРЭ}
        \addcontentsline{toc}{section}{1.3 Черты и принципы СОРЭ}
        \addcontentsline{toc}{section}{1.4 Роль государства в СОРЭ}
        \addcontentsline{toc}{section}{1.5 Недостатки СОРЭ}
        \addcontentsline{toc}{section}{2 Опыт, модели и направления СОРЭ в других странах}
        \addcontentsline{toc}{section}{2.1 Модели и направления СОРЭ}
        \addcontentsline{toc}{section}{2.2 СОРЭ в Швеции}
        \addcontentsline{toc}{section}{2.3 СОРЭ в Германии}
        \addcontentsline{toc}{section}{3 СОРЭ в РБ}
        \addcontentsline{toc}{section}{3.1 РБ в 90-е годы}
        \addcontentsline{toc}{section}{3.2 Основные социально-экономические показатели}
        \addcontentsline{toc}{section}{3.3 Социально-экономическое развитие в РБ}
        \addcontentsline{toc}{section}{3.4 Основы устойчивого развития до 2030 года}
        \addcontentsline{toc}{section}{3.5 Основные проблемы социально-экономического развития РБ}
        \addcontentsline{toc}{section}{ЗАКЛЮЧЕНИЕ}
        \addcontentsline{toc}{section}{Приложение А}
        \addcontentsline{toc}{section}{Приложение Б}
        \addcontentsline{toc}{section}{СПИСОК ИСПОЛЬЗОВАННЫХ ИСТОЧНИКОВ}
    \end{center}

    \newpage
    \begin{center}
        \textbf{\LARGE{ВВЕДЕНИЕ}}
    \end{center}
    \\
    \par
    Девяностые годы XX века были судьбоносными для экономики Беларуси. Распался Советский Союз, рухнули прежние принципы организации и функционирования экономики, практически все бывшие союзные республики оказались в глубоком политическом и экономическом кризисе.
    \par
    Именно становление экономической системы рыночного типа выдвигает в число первоочередных такие вопросы, как преобразование механизма управления, изменение принципов мотивации труда. Актуальность этих вопросов для Республики Беларусь вызвана сложностью переходного периода, выбором пути социально-экономического развития, кризисом всех сфер общества, необходимостью создания собственной модели экономики.
    \par
    Возникла необходимость в сжатые сроки создать собственную государственность, национальную банковскую систему, ввести национальную валюту, искоренить гиперинфляцию, насытить рынок элементарными потребительскими товарами, поддерживать отечественных производителей, сдерживать рост безработицы, защитить социально наиболее уязвимые слои населения, определить перспективную модель социально-экономического развития страны, овладеть навыками конкурирования на мировых рынках товаров и услуг.
    \par
    Актуальность темы исследования курсовой работы связана с необходимостью определения целей и условий формирования социально - ориентированной рыночной экономики, опираясь на опыт других стран, в определении проблем и перспектив социально-экономического развития Республики Беларусь. Анализ этих вопросов даст не только четкую оценку идеям, использовавшимся при формировании хозяйства нашей Республики, но и поможет определить и сформулировать новые, наиболее созвучные времени и мировым тенденциям развития экономики государства.
    \par
    Главной целью курсовой работы является необходимость понять условия формирования социально-ориентированной экономики, рассмотреть модель социально-ориентированной рыночной экономики в нашей стране, определить проблемы ее развития и соответствующие перспективы.
    \par
    Задачи курсовой работы:
    \begin{itemize}
        \item выявить понятие и сущность социально-ориентированной экономики;
        \item рассмотреть основные черты и модели социально-ориентированной экономики;
        \item проанализировать модель экономического развития Республики Беларусь и ее дальнейшие перспективы.
    \end{itemize}
    \par
    В ходе написания курсовой работы использовались учебные пособия, нормативные документы, статистические данные, интернет источники.

    \newpage
    \begin{center}
        \textbf{1 Социально-ориентированная рыночная(СОР) экономика}
        \\
        \HRule\\[0.5cm]
        \textbf{1.1 Определение и ключевые понятия}
    \end{center}
    \\
    \par
    Есть две противоположных экономических модели: плановая и рыночная экономика. Подавляющее большинство стран сейчас развиваются по рыночной модели.
    \par
    Социа́льно-ориенти́рованная ры́ночная эконо́мика (социа́льное ры́ночное хозя́йство) (нем. Soziale Marktwirtschaft) — экономическая система, организованная на основе рыночной саморегуляции, при которой координация действий осуществляется на основе взаимодействия на рынках свободных частных производителей и свободных индивидуальных потребителей. Модель социально-рыночной экономики исходит из требования, что ни государство, ни частный бизнес не вправе иметь полный контроль над экономикой, а должны служить людям. В этой разновидности смешанной экономики, так же как и в рыночной экономике, структура распределения ресурсов определяется исключительно решениями самих потребителей, поставщиков ресурсов и частных фирм. Однако при этом экономически более сильные обязаны поддерживать более слабых. Роль государства заключается в развитии чувства взаимной ответственности всех участников на рынке и в корректировке несправедливых тенденций в конкуренции, торговле и распределении доходов. Система рассматривалась как альтернатива «laissez-faire» капитализму и социализму.
    \par
    Концепцию начал реализовывать Людвиг Эрхард, министр экономики, а впоследствии федеральный канцлер Германии. Название системы социально-рыночной экономики дал в 1947 году экономист Альфред Мюллер-Армак. Важный вклад в развитие также внесли Франц Бёмruen, Вальтер Ойкен, Франц Оппенгеймер, Вильгельм Рёпке, Александер Рюстов, Жак Фреско, Концепция стала одним из существенных элементов идеологии христианско-демократического движения и получила поддержку со стороны социальных консерваторов, социальных либералов и социал-демократов.
    \par
    Термин «социально-ориентированная экономика» является в последнее время достаточно употребляемым, хотя в обыденном токовании представление о нем достаточно расплывчато. Не совсем понято, является ли ориентация на высокие социальные параметры характеристикой всех экономик, или она присуща только некоторым разновидностям. Как известно, любая экономическая система имеет своей целью достижение высокого уровня удовлетворения материальных и духовных потребностей людей, а потому обладает социальными моментами. Действительно, не может же экономика развиваться ради себя самой: конечным результатом и целью жизнедеятельности является человек. Тем не менее, под социально-ориентированной понимается такой тип экономической системы, который отвечает определенным критериям. Можно толковать социально-ориентированную экономику как систему, в которой наиболее значимой является перераспределительная функция государства, позволяющая устранить существенную дифференциацию доходов в обществе и ликвидировать полюса бедности и богатства. Данное определение будет не совсем точным и правильным, поскольку любое государство занимается перераспределением доходов посредством функционирования бюджетной системы, поэтому вряд ли можно говорить о социальности каждой экономики.
    \par
    Социализация экономики стала реальностью с 20 века. Многие исследователи признают ее настолько важной, что считают не только главной современной тенденцией, но и закономерностью общественного развития. Однако чтобы увидеть истоки социализации, следует заглянуть вглубь истории – 19 век. Тогда на фоне бурного развития капитализма и все более явственного проявления как его успехов, так и «острых углов» (нарастающего социального неравенства, безработицы, бедности, незащищенности широких слоев населения и т.д.) все острее и громче становится критика капитализма.[6, с.23]
    \par
    В русле такой критики появилось и развивалось особое направление теории, названное «социальная экономика». И хотя оно разделилось на два течения, основные его взгляды можно свести к следующему:
    \begin{itemize}
        \item критическая характеристика капитализма свободной конкуренции с точки зрения его социальных последствий: социальное неравенство, эксплуатация, безработица, угнетение женщин и т.п.;
        \item критическое отношение к частной собственности: от частичного до полного ее отрицания;
        \item отказ от идеи непримиримости и антагонизма классов, признания возможности их сотрудничества;
        \item возможность решения социальных проблем связывалась не с классовой борьбой, а с воздействием на социально-экономические процессы;
        \item признание необходимости активной политики государства социально-экономической области как особой его миссии.
    \end{itemize}
    \par
    Основные функции социально-рыночной экономики можно представить следующим образом. Если учесть, что общей функцией любой экономической системы является обеспечение материальных условий жизнеспособности общества, то она, естественно, остается главной функцией и для социально-рыночной экономики. Тогда другими, специфическими ее функциями будут:
    \begin{itemize}
        \item функция создания условий для более широкого удовлетворения потребностей социально-рыночной экономики
        \item функция социальной защиты тех, кто оказался в силу ряда объективных причин вне рыночной «игры» (дети, безработные и т.п.);
        \item повышения благосостояния всех слоев общества на основе экономического развития страны;
    \end{itemize}

    \begin{center}
        \textbf{1.2 Главные компоненты СОР экономики}
    \end{center}
    \\
    \par
    Главными составляющими компонентами социальной рыночной экономики является:
    \begin{itemize}
        \item рыночное хозяйство.
        \item социальная сфера.
    \end{itemize}
    \par
    Их соотношение, структура не могут заранее обозначиться жестко и строго, поскольку будут реально зависеть от конкретных условий (от уровня развития страны, ее экономического потенциала, истории, менталитета и т.п.), поэтому могут быть неодинаковыми в разных странах.
    \par
    Такая вариативность структуры социальной рыночной экономики заложена самой этой формулой, в которой «социальная» и «рыночная» сосуществуют на равных. В этой двойственности есть и плюсы и минусы. Минус состоит в недостаточной определенности и четкости, а плюс в том, что на базе этой формулы возможны варианты сочетаний, составляющих компонент. Главное, чтобы они были в рамках основополагающих принципов данной системы, не выходили из ее координат.
    \par
    Когда говорится о социальной ориентации, чаще всего подразумеваются западноевропейские государства, которые, хотя и не добились таких высот в развитии экономики, но основной целью функционирования общества выдвигают обеспечение социального диалога, партнерства, высокого уровня и качества жизни, социальной защищенности и бесконфликтности. С особой трепетностью граждане нашего государства традиционно относились к шведской модели, в которой реализовались аспекты, до сих пор не воплощенные в реальности нашего государства.
    \par
    В основу количественных критериев социально-ориентированной экономики могут быть положены следующие: место социальной политики среди приоритетов развития, распределение социальных функций между государством и иными субъектами хозяйствования, доля государственной собственности в экономической системе, объем перераспределяемого ВВП. [1]
    \par
    Использование подобного рода параметров позволяет выделить четыре модели:
    \begin{itemize}
        \item либеральную
        \item консервативную
        \item социал-демократическую
        \item рудиментарную
    \end{itemize}
    Либеральный режим (Великобритания) основывается на приоритете рыночного механизма, относительно низком объеме ВВП, перераспределяемом через налоговую и бюджетную систему (не более 40 процентов), осуществлении преимущественно пассивной политики на рынке труда, ориентированной на выплаты пособий по безработице, высоком удельном весе частных и общественных компаний в сфере производства частных услуг. Определенное государственное вмешательство в функционирование экономики несущественно изменяет условия жизни населения, которые обеспечивает рынок.
    \par
    Консервативный режим основан на рыночной логике распределения, относительно большой доле перераспределяемого ВВП (около 50 процентов), ориентации на поддержание полной занятости, развитой системе социального диалога и партнерства. Существенное внимание уделяется проблемам оказания социальной помощи и социального обеспечения. Страховые фонды формируются в основном за счет работодателей. Подобный режим развивается в Германии, Франции, Италии, Австрии и Бельгии.
    \par
    Социал-демократический режим реализуется посредством перераспределения 50–60 процентов производимого ВВП, активной политики на рынке труда, ориентации социальной политики на конкретного человека. Для экономики характерны высокий уровень дотаций и субсидий. Чрезмерные трансферты обеспечиваются высокими ставками налогообложения, что существенно снижает стимулы для предпринимательства и трудовой деятельности. В последнее время этот режим подвергается существенным модификациям. В качестве примеров можно определить Швецию, Данию, Норвегию и Финляндию.
    \par
    Рудиментарный режим характерен для наименее развитых стран региона, в которых высок уровень безработицы, объемы перераспределяемого ВВП могут существенно колебаться от 40 процентов (в Испании) до 60 процентов (в Греции). Социальная политика рассчитана на наиболее бедные категории граждан.
    \par
    Наибольший интерес для рассмотрения заслуживают шведская модель и современные попытки ее реформирования. Это обусловлено следующими моментами:
    \begin{itemize}
        \item Во-первых, здесь традиционно в качестве приоритетов социально-экономического развития рассматривают полную занятость и выравнивание доходов.
        \item Во-вторых, в данном государстве с 1932 г. (за исключением периода 1976–1982 гг.) у власти находятся социал-демократы. В-третьих, здесь
        \item В-третьих, представители данной нации очень близко воспринимают идею равенства.
    \end{itemize}

    \begin{center}
        \textbf{1.3 Черты и принципы СОР}
    \end{center}
    \\
    \par
    Основные черты модели социально-ориентированной рыночной экономики:
    \begin{itemize}
        \item обеспечение полной занятости населения;
        \item смешанная (государственно-частная) экономика. Одновременное наличие высокой доли государственной собственности и развитого института частной собственности;
        \item социальная безопасность, социальная справедливость и социальный прогресс (путём проведения государством мероприятий по перераспределению в форме оказания социальной помощи, социальных пенсий и уравнивающих платежей, субсидий, дотаций, прогрессивной шкалы подоходного налога и т. д., через систему социального обеспечения: пенсионное, медицинское страхование, страхование по безработице и по уходу, от несчастного случая, через трудовое и социальное законодательство);
        \item частная собственность на средства производства и свободное ценообразование;
        \item создание условий для конкуренции и обеспечение конкуренции (например, путём антимонопольного законодательства, законов против недобросовестной конкуренции);
        \item сознательная политика укрепления конъюнктуры экономического роста;
        \item политика стабильной валюты (в том числе через независимый эмиссионный банк);
        \item свобода внешней торговли, валютный обмен.
    \end{itemize}
    \par
    Модель социально-ориентированной рыночной экономики базируется на ряде основополагающих принципов:
    \begin{itemize}
        \item конституционные гарантии личных прав и свобод граждан;
        \item свобода предпринимательства и рыночной конкуренции;
        \item свободное ценообразование;
        \item возможность частной собственности на средства производства;
        \item равенство всех форм собственности (государственная, частная, акционерная, коллективная и т.д.);
        \item гарантии неприкосновенности частной собственности;
        \item независимый Центробанк, свободный валютный рынок, стабильная национальная валюта;
        \item низкий, близкий к нулю уровень инфляции;
        \item свободная внешняя торговля;
        \item низкие, близкие к нулю кредитные ставки;
        \item высокая ответственность бизнеса за своих работников;
        \item прямая зависимость благосостояния граждан от результатов их труда;
        \item высокий уровень социальной защиты нетрудоспособных и социально уязвимых категорий населения;
        \item социальное партнерство между государством, профсоюзами и союзами предпринимателей;
        \item низкий уровень коррупции;
        \item свободные выборы, сменяемость власти, независимая судебная система;
    \end{itemize}
    \par
    Таким образом, социально-ориентированная рыночная экономика, с одной стороны, смягчает такие присущие рыночной экономике признаки как высокий уровень социального неравенства,эксплуатация труда, высокая безработица, а с другой стороны, не подразумевает всеобщей уравниловки как плановая экономика, поскольку уровень благосостояния граждан напрямую зависит от результатов их труда. [2]

    \begin{center}
        \textbf{1.4 Роль государства в СОР}
    \end{center}
    \\
    \par
    Государство в данной рыночной модели играет роль регулятора, распределителя,
    можно сказать, арбитра, обеспечивающего справедливое взаимодействие между бизнесом и
    населением. Оно законодательно создает условия для свободного развития рыночной экономики и,
    одновременно, выступает гарантом защищенности интересов граждан, в том числе, наиболее
    социально-уязвимых слоев общества: пенсионеров, инвалидов, и т.д.
    \par
    Его главными целями являются:
    \begin{itemize}
        \item рост благосостояния граждан;
        \item финансирование базовых социальных услуг и развития инфраструктуры;
        \item экономическая стабильность;
        \item высокий уровень занятости населения;
        \item максимально благоприятные условия для населения и бизнеса.
    \end{itemize}
    \par
    При этом государство не противопоставляет интересы населения и бизнеса, социальную
    справедливость и экономическую эффективность, а пытается обеспечить наиболее выгодное
    взаимодействие этих двух групп, максимальный уровень компромисса между ними.
    Обеспечение оптимального уровня социального равенства и справедливости в
    социально-ориентированной рыночной экономике является не только политическим принципом
    (как в плановой экономике), а и сильным факторов достижения максимальной экономической
    эффективности. [3]
    \par

    \begin{center}
        \textbf{1.5 Недостатки СОР}
    \end{center}
    \\
    \par
    Всегда есть и обратная сторона - недостатки, давайте их тоже рассмотрим для объективности.
    \begin{itemize}
        \item Высокая социальная нагрузка на бюджет. Такая модель требует высокого
        уровня социальных расходов бюджета, что достигается за счет высокого уровня
        налогообложения бизнеса. В странах с социально-ориентированной рыночной экономикой
        всегда очень высокие налоги, что снижает эффективность ведения бизнеса и тормозит
        экономическое развитие.
        \item Раздутые штаты предприятий. За счет программ стимулирования занятости
        невозможно достичь оптимальных штатов сотрудников на предприятиях, зачастую они
        перенасыщены. А это, в свою очередь, тоже снижает производительность труда и
        рентабельность бизнеса.
        \item Расходы на поддержку курса национальной валюты.</strong> Данная экономическая
        модель предполагает обеспечение стабильного курса национальной валюты, а это требует
        огромных золотовалютных резервов для проведения валютных интервенций при необходимости.
        \item Безвозвратное расходование природных ресурсов. И, наконец, данная
        экономическая модель не способствует сохранению и восстановлению природных ресурсов.
        Проблема, конечно, достаточно глобальная, но и сюда ее тоже можно отнести. [4]
    \end{itemize}
    \par
    Ввиду этих недостатков, страны, использующие данную экономическую модель, как
    правило, имеют более низкие темпы прироста ВВП, чем страны с чисто рыночной моделью
    экономики.

    \newpage
    \begin{center}
        \textbf{2 Опыт СОРЭ в других странах}
        \\
        \HRule\\[0.5cm]
        \textbf{2.1 Модели и направления СОРЭ}
    \end{center}
    \\
    \par
    Существуют несколько основных направлений (моделей) социально-ориентированной рыночной
    экономики, каждое из которых имеет свои характерные особенности.
    \begin{itemize}
        \item Континентальная (Германская) модель.
        Для континентальной модели характерны очень высокие объемы перераспределения ВВП через
        госбюджет (около 50 процентов), высокоразвитая система социального партнерства, высокая занятость
        населения.
        Примеры стран с континентальной моделью социально-ориентированной рыночной экономики:
        \begin{itemize}
            \item Германия;
            \item Австрия;
            \item Бельгия;
            \item Нидерланды;
            \item Швейцария.
        \end{itemize}
        \item Англосаксонская модель.
        Для англосаксонской модели характерны чуть меньшие объемы перераспределения ВВП через
        госбюджет (около 40 процентов) и пассивная политика занятости.
        Примеры стран с англосаксонской моделью социально-ориентированной рыночной экономики:
        \begin{itemize}
            \item Великобритания;
            \item Ирландия;
            \item Канада.
        \end{itemize}
        \item Средиземноморская модель.
        Средиземноморская модель может характеризоваться разными уровнями перераспределения ВВП через
        госбюджет (от 40 до 60 процентов), а социальная политика здесь охватывает только самые уязвимые слои
        населения и не носит всеобъемлющего характера.
        Примеры стран со средиземноморской моделью социально-ориентированной рыночной экономики:
        \begin{itemize}
            \item Греция;
            \item Испания;
            \item Италия.
        \end{itemize}
        \item Скандинавская модель.
        Скандинавская модель отличается самым высоким уровнем перераспределения ВВП через госбюджет
        (около 50 \&- 60 процентов) и самой активной социальной политикой. В странах, использующих эту
        экономическую модель, благосостояние народа провозглашается как высшая цель экономического
        развития государства.
        Примеры стран со скандинавской моделью социально-ориентированной рыночной экономики:
        \begin{itemize}
            \item Швеция;
            \item Дания;
            \item Норвегия;
            \item Финляндия.
        \end{itemize}
    \end{itemize}
    \par
    Наиболее социально-ориентированными среди перечисленных моделей являются скандинавская и
    континентальная, а в качестве примеров наиболее успешных стран с
    социально-ориентированной рыночной экономикой всегда приводят скандинавские страны (в
    первую очередь Швецию) и Германию. К слову, именно эти страны традиционно
    занимают ведущие места в рейтинге индекса счастья(показателя, характеризующего
    общий уровень удовлетворенности жизнью населения). [5]

    \begin{center}
        \textbf{2.2 СОРЭ в Швеции}
    \end{center}
    \\
    \par
    Наибольший интерес для рассмотрения заслуживают шведская модель и современные попытки ее реформирования.
    Это обусловлено следующими моментами.
    Во-первых, здесь традиционно в качестве приоритетов социально-экономического развития рассматривают полную занятость и выравнивание доходов.
    Во-вторых, в данном государстве с 1932 г. (за исключением периода 1976–1982 гг.) у власти находятся социал-демократы.
    В-третьих, здесь очень сильную позицию имеют профсоюзы, которые оказывают существенное влияние на уровень и динамику заработной платы.
    В-четвертых, представители данной нации очень близко воспринимают идею равенства.
    \par
    Сам термин «шведская модель» используется в нескольких контекстах:
    \begin{itemize}
        \item для определения типа экономической системы, в которой существенное влияние на развитие и динамику рыночной экономики оказывает государство;
        \item для характеристики специфической ситуации на рынке труда, когда определенное место отводится системе проведения коллективных переговоров на основе активного участия профсоюзов;
        \item для определения активной политики на рынке труда и осуществления перераспределительной функции государства как приоритетов социально-экономического развития;
        \item как совокупность факторов социально-экономического и политического развития, которые обеспечивают достижение высокого уровня и качества
    \end{itemize}
    \par
    Основным элементом шведской социальной политики выступает социальное страхование.
    Его основная цель заключается в обеспечении граждан средствами в случае безработицы, болезни, необходимости получения медицинской помощи, рождения ребенка, по старости, в связи с несчастными случаями и травмами на производстве.
    Система страхования здоровья является средством создания условий для социально-экономического равенства.
    Она делает возможным получение медицинских услуг в случае необходимой неотложной медицинской помощи наравне с другими.
    Система социального страхования финансируется за счет налогов, взносов предпринимателей, трудящихся и неработающих по найму, доходов по процентам и вычетов из капитала различных фондов.
    Швеция явилась пионером во многих социальных начинаниях. Это касается, прежде всего, института социального партнерства, начало которому было положено в 1938 г., когда Шведская федерация профсоюзов и Шведская федерация работодателей подписали соглашение о мирном урегулировании трудовых конфликтов и необходимости заключения трудовых соглашений.
    Помимо этого, Швеция раньше всех пришла к необходимости и начала осуществлять активную политику на рынке труда, ввела запрет на строительство АЭС,
    стала перераспределять огромные средства через бюджет, а также выработала курс на построение общества всеобщего благоденствия.
    Несомненными достижениями шведского общества являются следующие:
    \begin{itemize}
        \item обеспечение высокого уровня жизни и социальных гарантий большей части населения общества без социальных потрясений и политических конфликтов;
        \item высокий уровень политической культуры, который позволил сформировать общественную систему диалога и кооперационный характер отношений между различными слоями населения;
        \item достижение высокого уровня социально-экономического развития и реализация таких важных экономических целей, как полная занятость, стабильный уровень цен, долговременный динамичный экономический рост;
        \item приоритет развития человеческого фактора, творческого начала в стимулировании трудовой деятельности, что нашло свое отражение в концепции «человеческого капитала». [2, с 154]
    \end{itemize}
    \par
    Анализ развития шведской экономики позволяет сделать вывод о том, что она построена в первую очередь на идеях кейнсианства относительно места и роли государства в экономической системе.
    Первая реформа экономики в этом государстве была проведена в условиях Великой депрессии 1930-х гг.
    Выход из создавшейся ситуации был найден за счет усиления государственного вмешательства в экономику, причем в Швеции с самого начала государство стало выполнять чрезмерные социальные функции.
    Основоположником шведской модели считается Г. Мюрдаль, который совершенно правильно обосновал связь развития техники и технологии с прогрессом социума, ибо все, что делается, целью своей имеет благо для человеческого общества.
    Стабильность в обществе была достигнута за счет компромисса между государством, предпринимателями и трудящимися, которые признали взаимные уступки друг другу. Рабочие отказались от проведения широкомасштабных политических акций, общенациональных забастовок и призывов к национализации собственности. Наниматели признали за государством право на осуществление социальных реформ. В результате сформировалась особая культура, в рамках которой все проблемы общества решались только мирным путем. Была достигнута, по сути, максимальная степень государственного вмешательства в рыночную систему. В период расцвета шведская модель характеризовалась разветвленной и комплексной системой социальной защиты населения.
    Более 50 процентов ВВП проходило через перераспределительные каналы, гарантировалось бесплатное образование, возможности получения бесплатных медицинских услуг, пособия по болезни или уходу за ребенком составляли 90 процентов заработка.
    Чрезмерные социальные выплаты тяжелым бременем легли на плечи предпринимателей и частного сектора экономики.
    Высокие ставки налогов перестали стимулировать инвестиционную активность, экономическая модель стала давать сбои.
    \par
    Кризис политической, экономической и социальной действительности стал проявляться в снижении или даже отсутствии темпов экономического роста, развертывании инфляционных процессов, росте безработицы.
    Шведская экономика, функционировавшая в условиях полной занятости, столкнулась с невиданным ранее уровнем безработицы.
    Наибольшего значения в 9.9 процентов он достиг в 1997 г. Выход был найден за счет сокращения государственного сектора в экономике, проведения консервативных преобразований.
    Это происходило накануне присоединения к ЕС. Шведское законодательство было приведено в соответствие с принципами Евросоюза.
    Реально это выразилось в существенном снижении ставок налогообложения, что не замедлило позитивно сказаться на экономике; сокращении масштабов и размеров социальных выплат, результатом чего стало углубление дифференциации в обществе.
    Де-регулирование экономики позволило повысить конкурентоспособность шведских товаров на мировом рынке, уменьшить долю традиционных и повысить долю наукоемких производств; возросла степень концентрации капитала.
    Экономическая система стала постепенно переходить к фазе оживления деловой активности, вследствие чего стала снижаться безработица.
    На современном этапе уровень последней стабилизировался в пределах полной занятости.

    \begin{center}
        \textbf{2.3 СОРЭ в Германии}
    \end{center}
    \\
    \par
    Если сравнивать шведскую экономику, например, с немецкой, то можно отметить определенное сходство и отсутствие элементов особой оригинальности.
    Несомненно, в начальный момент развития шведской модели определенные ее черты были новы и неповторимы, но затем другие государства стали повторять некоторые из них.
    Сегодня можно констатировать, что процессы развития социальной ориентации стран континентальной Европы и Скандинавского полуострова идут навстречу друг другу.
    Первые идут по пути формирования справедливого, высокоразвитого социально-ориентированного общества на наднациональном уровне, а скандинавским государствам пришлось несколько либерализировать свои экономики с целью придания им относительной гибкости, способности принимать и быстро адаптироваться к переменам, уменьшению перегруженности государственного бюджета социальными расходами, но сохранению при этом высокого уровня социальных гарантий в обществе.[6]
    \par
    Наиболее развитой страной европейского континента является Германия, которая по объему ВВП уступает только США и Японии.
    Объем ВВП на душу населения составляет в стране 23836 долл. в год (2003г.) (по паритету покупательной способности) Экономическая система современной Германии носит наименование «социального рыночного хозяйства».
    Она характеризуется достаточно существенной ролью государства в экономике, особенно в сравнении с США или Великобританией.
    Основу системы составляет деятельность государства, которое пытается осуществить перераспределение социальных благ между всеми членами общества; при выполнении своих функций оно опирается на крупные банки.
    Позиции данных финансовых институтов в экономике Германии с учетом фактического влияния на государство и бизнес оказываются существенно сильнее, чем в других государствах.
    \par
    Впервые термин «социальное рыночное хозяйство» и основные характеристики этого феномена были изложены в работе А. Мюллера-Армака «Регулирование экономики и рыночное хозяйство» 1947 г.
    Популярность новой теории, пытавшейся соединить рыночную экономику и социальную направленность системы, была высочайшей.
    Дело в том, что мировой кризис 1929–1933 гг. показал неэффективность саморегулирования рыночной экономики.
    В этой обстановке в среде экономистов стали складываться два подхода, каждый из которых претендовал на главенствующее положение в объяснении дальнейшего развития рынка: кейнсианство и неолиберализм.
    Сторонники первого обосновывали необходимость государственного регулирования экономики; другие пытались переосмыслить и усовершенствовать взгляды неоклассической школы применительно к новым реалиям.
    \par
    Современная немецкая модель характеризуется следующими чертами.
    Во-первых, важной чертой является индивидуальная свобода как условие функционирования рыночных механизмов и децентрализованного принятия решений.
    Это обеспечивается спецификой осуществляемых государством функций, среди которых следует выделить следующие:
    \begin{itemize}
        \item политика хозяйственного порядка, то есть обеспечение свободы деятельности хозяйствующих агентов на основе поддержания конкуренции и противодействия монополистической деятельности;
        \item поощрение малого и среднего бизнеса, развитие устойчивого и многочисленного среднего класса;
        \item социальная политика, направленная на защиту тех, кто по определенным причинам не может обеспечить себе достойное существование (инвалиды, пенсионеры, безработные), а также перераспределение доходов в обществе с целью устранения резких перекосов в распределении;
        \item экологическая политика, направленная на сохранение и поддержание окружающей среды;
        \item политика экономического роста и структурных сдвигов путем сглаживания конъюнктурных колебаний и ориентации на такие макроэкономические цели, как стабильность денежной единицы, полная занятость и равновесие платежного баланса.
    \end{itemize}
    \par
    Следующей чертой немецкой модели является защита и поощрение конкуренции.
    Чрезмерное внимание этому аспекту обусловлено тем, что Германия долгое время считалась «классической страной картелей»: их бурный рост в конце XIX – начале XX вв. приводил в смятение как либералов, считавших их угрозой индивидуальной свободе, так и социалистов, принимавших их за ростки нового общества.
    \par
    Третья черта рассматриваемой модели – перераспределение собственности и доходов.
    Как известно, в рыночной экономике доход рассматривается как плата за использование производственных факторов – труда (в статистике Германии принято говорить о «доходах от несамостоятельных видов деятельности») и капитала («доходы от предпринимательской деятельности и имущества», то есть прибыль, процент, арендная плата).
    \par
    Еще одной чертой рассматриваемой немецкой модели социального рыночного хозяйства является соучастие работников в управлении предприятием.
    Закон об уставе предприятия 1972 г. ввел институт советов предприятий, а Закон о соучастии в управлении 1974 г. обязал крупные предприятия до половины мест в наблюдательных советах предоставлять наемным работникам и их представителям.
    Таким образом, модель социального рыночного хозяйства получила важный момент стабилизации отношений в обществе, поскольку таким образом можно снимать противоречия между собственниками, управляющими и наемными работниками.
    \par
    Итак, социально ориентированная рыночная экономика-это высокоэффективная открытая экономика с развитым предпринимательством и рыночной инфраструктурой, действенным государственным регулированием доходов, заинтересовывающем предпринимателей в расширении и совершенствовании производства, а наемных работников в высокопроизводительном труде.
    Она гарантирует высокий уровень благосостояния добросовестно работающим членам общества, достойное социальное обеспечение нетрудоспособным (престарелым, инвалидам, женщинам, находящимся в отпуске по уходу за ребенком); эффективную охрану жизни, здоровья, прав и свобод всем гражданам. [7]
    \par
    Белорусская модель экономики схожа с немецкой и шведской по следующим принципам: социальное равенство, поощрение конкуренции, полная занятость, социальная политика, поощрение предпринимательства, профсоюзная политика, стимулирование трудовой деятельности.

    \newpage
    \begin{center}
        \textbf{3 СОРЭ в РБ}
        \\
        \HRule\\[0.5cm]
        \textbf{3.1 РБ в 90-е годы}
    \end{center}
    \\
    \par
    К настоящему времени сложилось несколько концепций осуществления трансформации командной экономики.
    Первая предполагает радикальные системные преобразования экономики и общества.
    Ее авторы опираются на монетаристскую концепцию, утверждающую, что преобразования должны происходить с минимальным участием государства.
    Основными направлениями преобразований по этой концепции являются либерализация цен, переход на свободное рыночное ценообразование, жесткое регулирование денежной массы, государственных кредитов и субсидий.
    Другие меры включают переход к конвертируемости отечественной валюты, приватизацию государственных предприятий и создание условий для появления новых негосударственных предприятий, демонополизацию промышленности, проведение реформы системы налогообложения, финансового сектора и государственной службы.
    Государство при таком варианте трансформации прямо не вмешивается в хозяйственную деятельность экономических субъектов, а создает условия для эффективного функционирования рыночной экономики.
    Это направление получило название «шоковая терапия».
    \par
    Шоковый вариант применяется для того, что достичь критической массы быстрых реформ и, закрепив их результаты, завоевать доверие населения.
    Он эффективен в случае необходимости быстрого преодоления тяжелого финансового положения и острого товарного дефицита в стране.
    Такой вариант предусматривает либерализацию цен, ведущую к инфляционному взрыву, и обычно вызывает резкое первоначальное ухудшение уровня жизни населения, и поэтому требует национального согласия.
    Стержнем экономической политики является борьба с инфляцией путем проведения жесткой денежно-кредитной политики.
    \par
    Сущность второго направления – постепенное создание рыночных отношений.
    Вторая концепция предусматривает длительное, эволюционное формирование рыночной экономики с сохранением многих структур.
    Авторов подобного варианта часто называют «градулистами».
    По их мнению, быстрый переход от административной экономики к рыночной просто невозможен.
    Кроме того, необходимо смягчить негативные последствия рыночных реформ, избежать массовой безработицы и резкого падения жизненного уровня населения.
    Основную роль в рыночных преобразованиях градулисты отводят государству.
    Наиболее характерный пример реализации второго направления – Китай и Венгрия, а среди стран СНГ – Беларусь.
    \par
    Именно Беларусь среди стран СНГ выделяется своей последовательностью и постепенностью в создании рыночных отношений.
    Весь период трансформации экономики Беларуси можно разделить на следующие этапы:
    \begin{itemize}
        \item 1991--1995гг.- период глубокого затяжного экономического кризиса;
        \item 1996--2000гг. - этап выхода экономики из кризисного состояния и углубления рыночных отношений;
        \item 2001--2005 гг. и следующие пятилетия - переход на инновационный путь устойчивого экономического развития.
    \end{itemize}
    \par
    До начала 90-х Беларусь не функционировала как целостная экономическая система, не имела национальной экономической политики.
    В результате ``общесоюзного разделения труда'' в БССР развивались ресурсо- и энергоемкие производства, зачастую не имевшие в республике ни сырьевой базы, ни потребителей, но в то же время позволявшие получать значительные преимущества и достаточное количество средств, чтобы стать высокоиндустриальной по социалистическим меркам страной.
    БССР в рамках централизованного планового хозяйства была ``сборочным цехом'', ``конечным элементом в технологической цеп''.
    Имевшее спрос на мировом рынке российское сырье здесь превращалось в неконкурентоспособную вне СЭВ продукцию.
    До середины 90-х годов в экономике страны наблюдался глубокий спад: росло число убыточных предприятий, снижались объемы производства в промышленности и сельском хозяйстве, сокращались бюджетные ассигнования, падал объем внутренних инвестиций.
    На предприятиях промышленности широко практиковалась неполная рабочая неделя, росла скрытая и явная безработица.
    К моменту выборов первого президента Беларуси экономическая ситуация в стране продолжала ухудшаться.
    В 1992--1994 гг. цены выросли в 32 раза, объем ВВП снизился на 20 процентов.
    В стране процветала коррупция. Собственность и власть сосредотачивались в руках государственной бюрократии, формировалась экономическая модель олигархического капитализма, подобная той, которая сложилась в России.
    В течение 1990--1994 гг. уровень жизни упал в 2--33 раза, значительно выросло социальное расслоение.
    \par
    Как следует из результатов социологического опроса, в 1994 г. за ``рынок'' выступало только 30,3 процентов опрошенных (в конце 1990 г. 62,69 процентов).
    Таким образом, готовность населения в конце 1990г. принять неведомые ему рыночные реформы к моменту президентских выборов была утеряна.
    \par
    Системные реформы в стране начали проводиться с 1993 года.
    Особенностью реформирования экономики республики являются значительные масштабы государственного регулирования этого процесса.
    Цены на основные продукты питания и другие товары первой необходимости регулируются государством.
    Подавляющее большинство промышленных предприятий находится в государственной собственности.
    Это позволило избежать развала промышленности и сдержать темпы роста безработицы.
    Однако эти реформы не произвели ожидаемого эффекта, экономическое положение страны ухудшалось.
    Дефицит госбюджета возрос с 1,6 процента ВВП в 1992 году до 8,3 процента в 1993-м.
    Республика оказалась перед лицом реальной угрозы гиперинфляции. [8]
    Одновременно углублялся экономический спад: ВВП Беларуси в 1992 году снизился на 9,6, а в 1993-м — на 7,6 процента; промышленное производство — соответственно на 9,4 и 10 процентов; капитальные вложения — на 29 и 15 процентов. (Приложение А: Таблица 1 – Основные экономические показатели Республики Беларусь, 1990--1994гг.)
    \\
    \par
    Ухудшалось положение и в социальной сфере: рос уровень безработицы, снижались реальные доходы населения.
    Нарастание кризисных явлений в белорусской экономике продолжалось и в 1994 году.
    \par
    Интенсивное развитие кризисных явлений в экономике страны было обусловлено отсутствием научно обоснованной модели перспективной хозяйственной системы.
    Реакцией на сложившуюся ситуацию в стране была разработка краткосрочных прогнозов и стабилизационных программ. Первые два прогноза (на 1993 и 1994 гг.) представляли собой экономическую программу правительства на очередной год.
    В дальнейшем разрабатывались прогнозы социально-экономического развития страны на последующие годы.
    Подготовка прогнозов осуществлялась в два этапа: первый - формирование концепции и второй - непосредственная разработка комплексного проекта. [9]
    \par
    Одновременно возникла необходимость в разработке стратегии социально- экономического развития Республики Беларусь, составлении долгосрочных прогнозов и программ, создании системы взаимоувязанных документов различного временного горизонта.
    Пришло осознание того, что без грамотно разработанной стратегии невозможно выработать эффективный тактический механизм хозяйствования.
    Это предопределило разработку в 1996 г. Национальной стратегии устойчивого развития Республики Беларусь до 2010 г, Основных направлений социально-экономического развития республики на 1996--2000 гг., в 1998 г. – Концепции социально-экономического развития страны до 2015 г., в 1999 г. – Основных направлений социально-экономического развития Республики Беларусь на 2001--2005 гг.
    \par
    Только с 1997 году ситуация начала кардинально меняться и в 2005 году сохранялась тенденция динамичного развития экономики страны.
    В основном были выполнены параметры прогноза социально-экономического развития Республики Беларусь на 2005 год, утвержденного Указом Президента Республики Беларусь от 10 сентября 2004 года № 437.
    \par
    Результаты функционирования экономики страны в 1996--2007 гг. достаточно впечатляющие.
    ВВП вырос в 2,3 раза, продукция промышленности – 3,0, инвестиции в основной капитал – в 3,7, реальные денежные доходы населения – в 4,7 раза.
    Уровень производства промышленной продукции докризисного периода (1990г.) был превышен в 2000 г., реальных денежных доходов населения – в 2001 г., ВВП – 2003 г., инвестиций в основной капитал – в 2006 г.
    В итоге за 1991--2007 гг. рост ВВП составил 151 процент, продукции промышленности – 185 процентов, инвестиций в основной капитал – 143, реальных денежных доходов населения – 239 процентов.
    Производство продукции сельского хозяйства вплотную приблизился к уровню 1990 г. – 99 процентов. (Приложение А: Таблица 2 – Основные экономические показатели Республики Беларусь, к 1990 г.)
    \\
    \par
    После распада СССР и связанного с ним экономического кризиса 1991--1995 гг. Республика Беларусь сумела преодолеть к 1996 г. спад производства и достигнуть в последующие годы положительной динамики макроэкономических процессов – обеспечить ежегодные приросты ВВП, промышленной и сельскохозяйственной продукции, стабилизировать положение на внутреннем рынке, укрепить экспортный потенциал.
    Была восстановлена система управления экономикой, отвечающая реалиям переходного периода, улучшена система социальной защиты населения по основным макроэкономическим показателям в 2001 г. достигнут уровень докризисного периода.
    В 2005 году ВВП по отношению к 1991 г.
    Составил 127 процентов, продукция промышленности - 153, сельского хозяйства – 90, реальные денежные доходы населения 178,8

    \begin{center}
        \textbf{3.2 Основные социально-экономические показатели}
    \end{center}
    \\
    \par
    Основные социально-экономические показатели на 2022 в Минске:
    \par
    Объем валового регионального продукта города в январе-августе 2022г. к уровню аналогичного периода 2021года в сопоставимых ценах составил 95 процентов.
    На долю г.Минска приходится 31,4 процент ВВП республики.
    \par
    За 8 месяцев текущего года предприятия г.Минска произвели 15,2 процентов республиканского объема промышленной продукции.
    Город специализируется на производстве грузовых автомобилей, тракторов, автобусов, велосипедов, телевизоров, трансформаторов, бытовых холодильников и морозильников, бытовых стиральных машин, шариковых и роликовых подшипников, шерстяных тканей, косметических средств для ухода за кожей, декоративной косметики, фармацевтических препаратов, керамических плиток.
    \par
    Объем промышленного производства в г.Минске в январе-августе 2022г. к уровню аналогичного периода 2021года в сопоставимых ценах составил 98,9 процентов, инвестиции в основной капитал – 80,2 процентов, розничный товарооборот – 94,5 процентов.
    \par
    Доля г.Минска во внешнеторговом обороте республики составила 30,2 процентов.
    За 7 месяцев 2022года экспорт товаров составил 4977,4млн. долларов США, импорт – 7256 млн. долларов. К уровню января-июля 2021г. экспорт товаров составил 81,2 процент, импорт–82,7 процента.
    Город экспортирует нефтепродукты, тракторы, грузовые автомобили и их части, холодильники и морозильники, телевизоры, лесоматериалы продольно- распиленные. [10]
    \par
    По данным статистики, основным источником формирования топливно\\-энергетических ресурсов (ТЭР)
    является импорт, который не только позволяет стране удовлетворять свои потребности, но и произвести товары, поставляемые на экспорт.
    Так, по итогам 2020 г. объем собственного производства (добычи) ТЭР на территории Беларуси составил 6,3 млн т условного топлива(т.у.т.) (в угольном эквиваленте), что составляет лишь 17 процентов от объема валового потребления за данный период (37 млн т.у.т.).
    В то же время объем импорта достиг 46,5 млн т.у.т.,что позволило не только удовлетворить внутренние потребности страны, но и продать часть ТЭР за ру-беж в объеме 15,4 млн т у. т.
    Таким образом, Беларусь остается страной, благополучие которой напрямую зависит от объема и условий поставки энергоносителей.
    Более 95 процентов импортируемого топлива приходится на нефть и природный газ с объемами покупки 16 млн т и 18,8 млн куб. м соответственно (данные 2020 г.).
    Как свидетельствуют цифры топливно-энергетического баланса, Беларусь не может обеспечить себя данными видами топлива за счет ресурсов,добываемых на территории самой страны, из-за чего данные позиции приобретают важнейшее, можно даже сказать критическое значение для белорусской экономики.
    Однако энергетический кризис РБ не грозит – в текущих условиях страна не только сохранила оговоренный объем поставок энергоносителей,но и использовала сложившуюся конъюнктуру для получения более выгодных условий поставок. (Приложение Б: Таблица 3 - Основные социально-экономические показатели)

    \begin{center}
        \textbf{3.3 Социально-экономическое развитие в РБ}
    \end{center}
    Социально-экономическое развитие – это процесс социально-экономического развития общества.
    Социально-экономическое развитие измеряется такими показателями, как ВВП, ожидаемая продолжительность жизни, грамотность и уровень занятости.
    Ежегодно происходит планирование и прогноз социально-экономического развития страны.
    Документ издается в целях обеспечения стабильности в обществе и роста благосостояния граждан и определяет важнейшие параметры прогноза социально-экономического развития страны на грядущий год.
    Правительство в двухмесячный срок утверждает целевой план социально-экономического развития Беларуси на грядущий год, обеспечивающий полную мобилизацию ресурсов экономики для достижения поставленных, обычно, Всебелорусским народным собранием целей.
    При этом должны быть соблюдены все параметры, обеспечивающие макроэкономическую сбалансированность.
    \par
    Президент Беларуси Александр Лукашенко 9 декабря подписал Указ № 481 ``О важнейших параметрах прогноза социально-экономического развития Республики Беларусь на 2022 год''.
    Документ издан в целях обеспечения стабильности в обществе и роста благосостояния граждан и определяет важнейшие параметры прогноза социально-экономического развития страны на 2022 год.
    В частности, рост валового внутреннего продукта запланирован в размере 102,9 процентов к уровню текущего года, реальных располагаемых денежных доходов населения - 102 процентов, инвестиций в основной капитал - 103,3 процентов.
    Ожидается, что экспорт товаров и услуг вырастет на 6,3 процента к уровню 2021 года.
    Национальному банку совместно с Советом Министров поручено принять необходимые меры по ограничению в 2022 году инфляции на уровне 6 процентов.
    Правительство в двухмесячный срок утвердит целевой план социально-экономического развития Беларуси на 2022 год, обеспечивающий полную мобилизацию ресурсов экономики для достижения поставленных VI Всебелорусским народным собранием целей.
    При этом должны быть соблюдены все параметры, обеспечивающие макроэкономическую сбалансированность. [11]
    \par
    Программой определены цели, задачи и приоритетные направления социально-экономического развития страны, ключевые меры по их реализации, отражены ожидаемые результаты и целевые индикаторы развития отраслей, сфер экономики и регионов.
    Основные приоритеты ее реализации:
    \begin{itemize}
        \item рост благосостояния граждан;
        \item обеспечение комфортного проживания в каждом регионе страны;
        \item развитие человеческого потенциала.
    \end{itemize}
    \par
    11 октября 2022 г. Международный валютный фонд опубликовал обзор мировой экономики (World Economic Outlook), в котором представил свои прогнозы.
    Аналитики МВФ отмечают, что мировая экономика сталкивается с целым рядом вызовов, влияющих на неопределенность будущего.
    Речь, прежде всего, идет о высокой инфляции, что заставляет центробанки наиболее развитых стран мира переходить к ужесточению денежно-кредитной политики и сворачиванию фискальных стимулов,которые вводились для смягчения последствий COVID-19.
    Существен-ную неопределенность в европейском регионе создает российско-украинский конфликт, продолжающий влиять на поставки важных для региона товаров.
    Сохраняются сбои в цепочках поставок – из-за ограничений, связанных как с военными действиями, так и с пандемией COVID-19.
    В итоге МВФ сохранил свой июльский прогноз по замедлению роста мировой экономики с 6 процента в 2021 г. до 3,2 процента в 2022 г. и снизил ожидание на 2023 год с 2,9 процента до 2,7 процента.
    Прогнозируется, что мировая инфляция вырастет с 4,7 процента в 2021 г. до 8,8 процентов в 2022 г., но снизится до 6,5 процентов в 2023 г. и до 4,1 процента к 2024 г.
    Ухудшил МВФ свои прогнозы и в отношении белорусской экономики – в текущем году аналитики фонда ожидают падения ВВП на 7 процентов вместо ранее озвученных 6,4 процента.
    По поводу перспектив на 2023 год МВФ также стал более пессимистичным: прогнозируется мизерный рост ВВП на 0,2 процента вместо его подъема на 0,4 процента, как предсказывалось в апреле 2022 г.
    Таким образом, МВФ ухудшил свои прогнозы в отношении РБ и на текущий, и на следующий год. Что касается остальных макроэкономических показателей, МВФ ожидает годовую инфляцию в Беларуси по итогам 2022 г. на уровне 16,5 процентов.
    В следующем году рост потребительских цен сохранит двузначные значения и может достичь 13,1 процентов.
    Дефицит баланса текущего счета в этом и следующем году составит более 1 процент ВВП, а существенного роста безработицы не предвидится.
    Приходится констатировать, что аналитиков МВФ не впечатлили ни достижения белорусского правительства по сокращению темпов падения ВВП, ни меры по обузданию роста цен, ни успехи во внешней торговле. [11]
    \par
    Запрет на повышение цен, введенный в стране, вызывает множество вопросов со стороны бизнеса.
    Один из самых частых – как долго этот запрет сохранится?
    Запрет на рост цен — это попытка сдерживать инфляцию, которая находится на повышенном уровне.
    Официальный прогноз на 2022 год предполагал рост потребительских цен на 6 процентов, однако по факту с начала года они уже увеличились на 14,8 процентов, а в годовом выражении (за последние 12 месяцев) – на 17,4 процентов.
    Текущий уровень инфляции намного выше показателей, которые фиксировались в предыдущие годы.
    Когда, в 2021 г. потребительские цены в стране выросли на 9,97 процентов, в 2020-м инфляция составила 7,4 процентов, в 2019-м – 4,7 процента, в 2018-м – 5,6 процентов, в 2017-м – 4,6 процента, в 2016-м – 10,6 процентов, в 2015-м – 12 процентов.
    С 2015 г. для снижения инфляционных процессов в стране Нацбанк перешел на режим монетарного таргетирования.
    Суть его сводится к тому,что снижение инфляции обеспечивается за счет ограничения денежного предложения.
    Благодаря фактическому отказу от бурного эмиссионного финансирования экономики (которое использовалось в прошлом) и удалось сбить инфляцию настолько, что в 2017–2019 гг. она находилась вблизи исторических минимумов.
    Таким образом, рыночное регулирование, основанное на монетарном таргетировании, оказалось достаточно эффективным средством для борьбы с инфляцией в Беларуси.
    Однако в 2020–2022 гг. потребительские цены в РБ вновь стали неуклонно расти. Причина оказалась в том, что наша страна в этот период столкнулась с несколькими вызовами.
    Ускорение инфляции в 2020 г. стало следствием заметного ослабления курса рубля к иностранным валютам.
    Кроме того, цены росли из-за неурожая в регионе отдельных сельхозкультур (подсолнечника и сахарной свеклы).
    Продолжение роста цен в 2021 г. связано с ростом мировых цен на продукты питания.
    Дополнительный вклад в увеличение цен обеспечил низкий урожай ряда сельскохозяйственных культур.
    В 2022 г. тренд на ускорение инфляции, к сожалению, сохранился.
    Правительство объясняет это, прежде всего, ослаблением белорусского рубля к российскому.
    С начала года курс российского рубля в Беларуси вырос более чем на 20 процентов.
    Влияние курса российского рубля на внутренние цены Беларуси обусловлено высокой долей импорта товаров из России.
    [12]
    \par
    Базовым условием достижения целей является устойчивый качественный рост экономики.
    Обеспечение стабильности в обществе и рост благосостояния граждан за счет модернизации экономики, наращивания социального капитала, создания комфортных условий для жизни, работы и самореализации человека – это цель развития страны на 2021–2025 годы, определенная Программой.
    Основными задачами текущего пятилетия являются:
    \begin{itemize}
        \item обеспечение роста ВВП (не менее чем в 1,2 раза в реальном выражении к уровню 2020 года);
        \item укрепление здоровья нации (повышение ожидаемой продолжительности жизни до 76,5 года в 2025 году);
        \item увеличение реальных располагаемых денежных доходов населения (в 1,2 раза, включая темп роста размеров пенсий выше уровня инфляции);
        \item создание условий для привлечения «длинных» денег в экономику, обеспечение роста инвестиций (более чем в 1,2 раза к уровню 2020 года);
        \item увеличение экспорта товаров и услуг более чем на 50 млрд. долларов США в 2025 году, а также диверсификация его структуры;
        \item повышение конкурентоспособности производственного сектора экономики, в том числе путем создать новые высокотехнологичные производства;
        \item увеличение доли сферы услуг в ВВП до 50–51 процента (в том числе ускоренное развитие наукоемких высокотехнологичных услуг);
        \item повышение качества образования и развитие новых профессиональных компетенций в соответствии с потребностями экономики;
        \item обеспечение устойчивости бюджетной системы, развитие финансового рынка;
        \item создание комфортной среды проживания и новых рабочих мест, гарантирующих достойную оплату за эффективный труд.
    \end{itemize}
    Внедрение информационно-коммуникационных и передовых производственных технологий во все сферы жизнедеятельности. Целевым ориентиром является доля сектора информационно-коммуникационных технологий в ВВП страны – не менее 7,5 процента в 2025 году.
    достижение уровня безработицы в трудоспособном возрасте (по методологии Международной организации труда) – не более 4,2 процента к концу 2025 года. [13]
    \par
    Ввиду санкционного режима многие логистические пути удлинились.
    Сырье и ресурсы стали дороже, соответственно подорожал и конечный продукт.
    Также я бы хотел отметить:
    \begin{itemize}
        \item уменьшение доли малообеспеченного населения (с доходами ниже бюджета прожиточного минимума) до 4,5 процентов;
        \item совершенствование существующей пенсионной системы;
        \item Повышение качества и доступности образования.
    \end{itemize}
    \par
    В целях обеспечения независимости в Беларуси созданы условия для того, чтобы транзакции по всем платежным инструментам при необходимости обрабатывались на территории страны.
    И еще одна антисанкционная мера – ориентация Беларуси на международные платежные системы из политически дружественных стран(России, Китая) усиливается.
    В планах государства развитие взаимодействия с платежными системами «Мир» и UnionPay. [14]
    \par
    В отличие от предыдущих семи
    лет, бюджет на будущий год формируется исходя из целевого сценария развития экономики с ростом ВВП на 3,8 процента.
    Как отметил Р. Головченко, это напряженная задача, которая создает определенные риски для устойчивости бюджета к внешним воздействиям.
    Главная цель, подчеркнул премьер-министр, остается прежней – обеспечение социальных гарантий граждан и финансирование важнейших государственных расходов.
    Конкретных цифр по бюджету озвучено немного.
    Известно, что в соответствии с заданием на рост оплаты труда на эти цели будет дополнительно выделено 3 млрд руб., из которых 2 млрд руб.
    пойдут на добавки к зарплате бюджетникам (без госслужащих и военных).
    Это позволит на 10 процентов увеличить базовую ставку и еще дополнительно предусмотреть средства на целевые выплаты.
    Как отметил министр финансов Юрий Селиверстов, по предварительным прогнозам, целевые выплаты и надбавки прирастут где-то на 50 процентов.
    По словам министра, в следующем году объем инвестиций, дорожного строительства, ремонта уличной сети, инфраструктуры к жилью будет не меньше текущего года.
    Увеличение расходов бюджета при снижении доходов приведет к тому, что главный финансовый документ на 2023 год будет сформирован с дефицитом в пределах 3–4 млрд руб. при плановом дефиците бюджета в 2022 г. 2,8 млрд руб.
    Одновременно Ю.Селиверстов заверил, что дефицит будет обеспечен ресурсами для того, чтобы профинансировать государственные расходы в полном объеме.
    Рассказал министр и об исполнении бюджета в текущем году.
    По итогам 9 месяцев фактический дефицит составил около 2,3 млрд руб. при плановом значении 2,8 млрд.
    Это свидетельствует о том, что бюджет исполняется в пределах запланированных объемов.
    По словамчиновника, особых проблемных вопросов не возникает, все расходы, предусмотренные бюджетом, финансируются в полном объеме.
    [15]
    \par
    Исходя из этого можно сделать следующие выводы:
    \begin{itemize}
        \item Прогноз социально-экономического развития Республики Беларусь происходит ежегодно;
        \item Выполнению некоторых планов из социально-экономического развития на 2022 может помешать санкционный режим.
    \end{itemize}

    \begin{center}
        \textbf{3.4 Основы устойчивого развития до 2030 года}
    \end{center}
    \\
    \par
    В 2018 году в соответствии с Законом Республики Беларусь «О государственном прогнозировании и программах социально-экономического развития Республики Беларусь» начинается новый пятилетний цикл разработки прогнозов и программ. Исходя из его положений во
    главе пирамиды прогнозно-плановых документов стоит Национальная стратегия устойчивого
    социально-экономического развития Республики Беларусь (НСУР) на 15 лет.
    \par
    Национальная стратегия устойчивого развития (НСУР) – это документ, определяющий направления стабильного развития трех взаимосвязанных и взаимодополняющих компонент: человек как личность и генератор новых идей – конкурентоспособная экономика – качество окружающей среды в условиях внутренних и внешних угроз и вызовов долгосрочного развития.
    НСУР-2030 – это долгосрочная стратегия, определяющая цели, этапы и сценарии перехода Республики Беларусь к зрелому гражданскому обществу и инновационному развитию экономики при гарантировании всестороннего развития личности, повышении стандартов жизни человека и обеспечении благоприятной окружающей среды. [16]
    Были проанализированы следующие компоненты:
    \begin{itemize}
        \item Социальный компонент («развитие человека»)
        - По индексу человеческого развития- 53-е место из 187 стран --- с высокий уровень человеческого развития
        Риски и возможные угрозы:
        \begin{itemize}
            \item процесс старения населения;
            \item рост демографической нагрузки;
            \item ухудшение качества здоровья граждан;
            \item недостаточная общественная активность граждан.
        \end{itemize}
        \item Экономический компонент устойчивого развития
        - Экономика Республики Беларусь активно развивается и интегрируется в мировое хозяйство, наращивая объемы экспорта и расширяя географию стран-партнеров
        \item Экологический компонент («окружающая среда»)
        - за 2005–2012 годы показывает наличие устойчивой тенденции к снижению техногенной нагрузки на единицу ВВП.
        Экологичность развития достигается посредством мер, обеспечивающих снижение антропогенной нагрузки на компоненты природной среды, уменьшение объемов образования отходов (в том числе токсичных) и предотвращение их вредного воздействия на окружающую среду и здоровье граждан, максимальное вовлечение отходов в гражданский оборот в качестве вторичного сырья, а также максимальное использование возобновляемых ресурсов и др.
    \end{itemize}
    \par
    Развитие мировой экономики в начале ХХI века будет предопределяться следующими тенденциями:
    \begin{itemize}
        \item усиление глобализации, международной интеграции;
        \item усиление конкуренции на мировых рынках и глубины дифференциации стран по уровню экономического развития;
        \item нарастание мировых миграционных процессов;
        \item возрастание роли человеческого капитала;
        \item ускорение темпов научно-технологического прогресса;
        \item истощение мировых запасов природно-сырьевых ресурсов;
        \item усиление значимости и влияния экологической компоненты на динамику экономического роста.
    \end{itemize}
    Основные вызовы устойчивому развитию мировой экономики:
    \begin{itemize}
        \item усиление конкурентной борьбы за факторы производства: человека, инновации, энергетические и сырьевые ресурсы, продовольствие и чистую воду;
        \item сокращение численности и низкий уровень воспроизводства населения в европейских странах на фоне быстрого роста населения в развивающихся, что усиливает территориальные диспропорции на миром рынке труда. Демографические проблемы затронули и Беларусь, усугубив ситуацию с нехваткой кадров, прежде всего в сельских районах;
        \item угроза энергетической безопасности в мировом масштабе. Все государства увеличивают инвестиции в альтернативные и «зеленые» энергетические технологии. Возведение в настоящее время Белорусской АЭС только частично снимет решение данной проблемы;
        \item угроза глобального изменения климата и уменьшения биоразнообразия, риск нарушения экологического равновесия и водного баланса территорий;
        \item наращивание производственного потенциала сверх экологической емкости территорий, что порождает угрозу экологической безопасности;
        \item проблема продовольственной безопасности в мировом масштабе. Вместе с тем для Беларуси этот мировой вызов открывает новые возможности для наращивания экспорта производимого в стране продовольствия.
    \end{itemize}
    \par
    Ожидаемые результаты следующие:
    \begin{itemize}
        \item увеличение ожидаемой продолжительности жизни при рождении до 77 лет;
        \item рост ВВП за 2016--2030 годы в 1,5–2,0 раза;
        \item достижение ВВП на душу населения к 2030 году 30-–39 тыс. долл. США по ППС (против 17,6 в 2013 году);
        \item повышение затрат на научные исследования и разработки – до 2,5 процента от ВВП в 2030 году;
        \item рост удельного веса затрат на охрану окружающей среды – до 2--3 процентов к ВВП в 2030 году;
        \item сокращение объемов выбросов парниковых газов на 15 процентов по сравнению с 1990 годом.
    \end{itemize}
    \par
    Данный документ, показывает, что Республика Беларусь намерена как развиваться и самостоятельно, так и укреплять свои позиции на внешних рынках.

    \newpage
    \begin{center}
        \textbf{3.5 Основные проблемы социально-экономического развития РБ}
    \end{center}
    \\
    \par
    Опросы белорусских предприятий свидетельствуют, что санкции вошли в ТОП-3 барьеров, которые препятствуют бизнесу успешно развиваться.
    Поэтому неудивительно, что крупные компании в наибольшей мере столкнулись в этом году с высокой неопределенностью.
    \par
    Опросы бизнеса позволяют выяснить положение и ожидания субъектов хозяйствования.
    Поскольку от них зависит развитие экономики, власти многих стран заказывают организацию таких анкетирований. [17]
    В последние годы в европейских странах для оценки экономического положения предприятий стал использоваться новый показатель – индекс неопределенности.
    Индикатор рассчитывается на основании ответов руководителей предприятий о том, насколько им сложно строить прогнозы (планы) относительно будущей деятельности.
    Измерение этого показателя тестировалось в некоторых странах ЕС в 2019–2021 гг., а с прошлого года он стал обязательным компонентом программы обследований бизнеса в Евросоюзе.
    Актуальность нового индикатора стала очевидной для многих из-за пандемии COVID-19, породившей высокую неопределенность в мире.
    Соответственно, возникла потребность в бизнес-индикаторе, который позволял бы следить за динамикой неопределенности.
    Для изучения экономической ситуации в Беларуси данный показатель также важен, поскольку не только пандемия, но и санкции привели к тому, что бизнес-перспективы наших предприятий становятся все более туманными.
    В июле этого года впервые индекс неопределенности стал рассчитывать Научно-исследовательский экономический институт Минэкономики, который на протяжении уже многих лет проводит в Беларуси опросы предприятий.
    Результаты измерения индекса неопределенности опубликованы в материалах XXIII Международной научной конференции, состояв-шейся в третьей декаде октября в Минске.
    Индекс рассчитан на основе ответов предприятий о том, насколько им сложно планировать (прогнозировать) свою деятельность.
    Чем выше названный показатель, тем у большего количества респондентов(компаний) возникают проблемы при прогнозировании своей деятельности. [18]
    \par
    Во-первых, что географическое расположение белорусских субъектов хозяйствования несущественно влияет на индекс неопределенности.
    А вот размер компании имеет значение.
    Так, наибольший индекс неопределенности (50) присущ предприятиям-гигантам (с численностью свыше 5000 работников).
    Наименьшее значение индекса (27)зафиксировано в компаниях с численностью работников от 101 до 250 человек.
    Во-вторых, весьма значительно индекс неопределенности отличается в отраслевом разрезе.
    Как выяснилось, наивысшая неопределенность характерна для химической промышленности (значение индекса 59).
    Ни один из руководителей предприятий химической отрасли не выбрал ответ, что прогнозы делать легко.
    Намного лучше выглядит ситуация в отраслях, ориентированных на физических лиц и удовлетворяющих их базовые потребности.
    Именно таким предприятиям, которые обеспечены заказами, легче планировать свою дальнейшую деятельность.
    Как следствие, для предприятий фармацевтической отрасли характерен наименьший (21) индекс неопределенности.
    Низкое значение данного бизнес-индикатора характерно и для производителей продуктов питания. [19]
    \par
    Сложности с бизнес-планированием (прогнозированием) у предприятий нередко появляются из-за возникающих на пути их развития барьеров.
    Результаты опросов промышленных предприятий, проведенных НИЭИ Минэкономики в первом полугодии 2022 г., опубликованные в материалах вышеупомянутой конференции, позволяют сделать несколько выводов.
    По-прежнему главным барьером на пути развития промышленных компаний являются высокие цены на сырье и материалы – на это указали 58 процентов опрошенных предприятий (против 55 процентов в первой половине прошлого года).
    Вторым по значимости фактором, сдерживающим развитие промышленных предприятий, в 2022 г. впервые оказались санкции (на значимость этого барьера указали 41 процент респондентов).
    Из-за санкционного давления ощутимо возросла проблема дефицита необходимого сырья и полуфабрикатов.
    Если в первой половине прошлого года на это препятствие обращали внимание 14 процентов респондентов, то в первом полугодии 2022 г. – 22 процента руководителей опрошенных компаний.
    Помимо санкций в ТОП-5 самых значимых барьеров для развития бизнеса попали и другие факторы.
    В частности, 39 процентов респондентов обратили внимание на низкий платежеспособный спрос, 35 процентов – на недостаток оборотных средств, 34 процента – на неплатежи потребителей.
    По сравнению с прошлым годом значимость перечисленных факторов для промышленных предприятий несколько снизилась.
    Также, к счастью, эпидемиологическую обстановку субъекты хозяйствования стали воспринимать более позитивно.
    Если в первой половине прошлого года коронавирусная инфекция сдерживала развитие 27 процентов опрошенных предприятий, то в 2022 г. на существенность этого фактора указали только 12 процентов респондентов.
    Как видим, несмотря на некоторое смягчение отдельных проблем, многие из них сохраняют свою значимость для предприятий.
    К традиционным факторам, препятствующим развитию бизнеса, в этом году добавились западные санкции.
    \par
    Поскольку предсказать масштаб и длительность санкционного давления крайне сложно, субъекты хозяйствования, особенно под-санкционные, столкнулись с труд-ностями планирования своей будущей деятельности.
    На этом фоне появление нового бизнес-индикатора – индекса неопределенности – можно только приветствовать, поскольку он позволяет измерять, как изменяются возможности предприятий планировать свою будущую деятельность.
    [20]
    \par

    \newpage
    \begin{center}
        \textbf{\LARGE{ЗАКЛЮЧЕНИЕ}}
    \end{center}
    \\
    \par
    В данной работе в соответствии с поставленными целями и задачами была исследована социально-ориентированная рыночная экономика, ее история и примеры на различных странах, включая Республику Беларусь, выявлены ее особенности, а также были рассмотрены пути совершенствования.
    Обобщая анализ каждого из рассмотренных вопросов, можно сделать следующие выводы:
    \begin{enumerate}
        \item Социально-ориентированная рыночная экономика-это высокоэффективная открытая экономика с развитым предпринимательством и рыночной инфраструктурой, действенным государственным регулированием доходов, заинтересовывающем предпринимателей в расширении и совершенствовании производства, а наемных работников в высокопроизводительном труде. Она гарантирует высокий уровень благосостояния добросовестно работающим членам общества; достойное социальное обеспечение нетрудоспособным (престарелым, инвалидам, женщинам, находящимся в отпуске по уходу за ребенком); эффективную охрану жизни, здоровья, прав и свобод всем гражданам.
        \item Социально-экономическое развитие – это процесс социально-экономического развития общества.
        \item Социально-экономическое развитие измеряется такими показателями, как ВВП, ожидаемая продолжительность жизни, грамотность и уровень занятости.
        \item Ежегодно происходит планирование и прогноз социально-экономического развития страны.
        \item Правительство в двухмесячный срок утверждает целевой план социально-\\экономического развития Беларуси на грядущий год, обеспечивающий полную мобилизацию ресурсов экономики для достижения поставленных, обычно, Всебелорусским народным собранием целей.
        \\
        При этом должны быть соблюдены все параметры, обеспечивающие макроэкономическую сбалансированность.
    \end{enumerate}

    \newpage
    \begin{center}
        \textbf{\LARGE{ПРИЛОЖЕНИЕ А}}
    \end{center}

    \newpage
    \begin{center}
        \textbf{\LARGE{ПРИЛОЖЕНИЕ Б}}
    \end{center}

    \newpage
    \setcounter{page}{35}
    \begin{center}
        \renewcommand\refname{СПИСОК ИСПОЛЬЗОВАННЫХ ИСТОЧНИКОВ}
        \begin{thebibliography}{10}
            \bibitem{1_wiki1} Социальное рыночное хозяйство: Теория и этика экономического порядка в России и Германии. / Пер. с нем. под ред. В. С. Автономова. — СПб. : Экономическая школа, 1999. — 367 с. — (Этическая экономия: исследования по этике, культуре и философии хозяйства; Вып.6). — ISBN 5-900428-43-5
            \bibitem{2_wiki2} Давыдова Т. Е. Формирование и историческое развитие концепции социального рыночного хозяйства // Историко-экономические исследования. 2006. № 1.
            \bibitem{3_wiki3} Социальное рыночное хозяйство: концепция, практический опыт и перспективы применения в России / Под ред. Р. М. Нуреева. — М.: Издательский дом ГУ-ВШЭ, 2007.
            \bibitem{4_wiki4} Социальное рыночное хозяйство — основоположники и классики : сборник научных трудов / авт. предисл. К. Кроуфорд; ред.-сост. К. Кроуфорд, С. И. Невский, Е. В. Романова и др. — М. : Весь Мир, 2017. — 418 с. : ил. — ISBN 978-5-7777-0676-8
            \bibitem{5_swe-volkov} Волков, А. М. Швеция: социально-экономическая модель / А. М. Волков. - М.: Мысль, 1991. – 188 c.
            \bibitem{6_indexHap} Индекс счастья - https://fingeniy.com/indeks-schastya/
            \bibitem{7_wiki5} История концепции социального рыночного хозяйства в Германии / Ред.-сост.: С. И. Невский, А. Г. Худокормов. — М.: ИНФРА-М, 2022. — 212 с. ISBN 978-5-16-017090-9
            \bibitem{8_wiki6} Социальное рыночное хозяйство / С. И. Невский, А. А. Ткаченко // Большая российская энциклопедия : [в 35 т.] / гл. ред. Ю. С. Осипов. — М. : Большая российская энциклопедия, 2004—2017.
            \bibitem{9_ec2003} Шимов, В.Н. Экономическое развитие Беларуси на рубеже веков: проблемы, итоги, перспективы: Моногр. / В.Н. Шимов. - Мн.: БГЭУ, 2003. – 229 с.
            \bibitem{10_belstat2022-addon} Статистический сборник Беларусь в цифрах, 2022
            \\
            https://www.belstat.gov.by/ofitsialnaya-statistika/publications/izdania/
            \\
            \bibitem{11_https://president.gov.by} Утверждены важнейшие параметры прогноза социально-экономического развития Беларуси на 2022 год
            \\
            https://president.gov.by/ru/events/utverzhdeny-vazhneyshie-parametry-prognoza-socialno-ekonomicheskogo-razvitiya-belarusi-na-2022-god
            public\_compilation/index\_50202/
            \bibitem{12_etalone.by/25} Социально-экономическое развитие: определены направления до 2025 года
            \\
            https://etalonline.by/novosti/korotko-o-vazhnom/sotsialno-ekonomicheskoe-razvitie-do-2025-goda/
            \bibitem{13_etalone.by/22} О важнейших параметрах прогноза социально-экономического развития Республики Беларусь на 2022 год
            \\
            https://etalonline.by/novosti/korotko-o-vazhnom/sotsialno-ekonomicheskogo-razvitiya-respubliki-belarus/
            \bibitem{14_belstat2022} Социально-экономическое положение Республики Беларусь
            \\
            https://www.belstat.gov.by/ofitsialnaya-statistika/izdania/\_bulletin/inde\_57418/
            \bibitem{15_neg66} Экономическая Газета / Номер 66 - https://neg.by
            \bibitem{16_neg74} Экономическая Газета / Номер 74 - https://neg.by
            \bibitem{17_neg78} Экономическая Газета / Номер 78 - https://neg.by
            \bibitem{18_neg79} Экономическая Газета / Номер 79 - https://neg.by
            \bibitem{19_neg80} Экономическая Газета / Номер 80 - https://neg.by
            \bibitem{20_neg84} Экономическая Газета / Номер 84 - https://neg.by
        \end{thebibliography}
    \end{center}
\end{document}